\section{Движение электронов во внешнем магнитном поле}\label{Ch:Fermion}
\subsection{Волновые функции электронов во внешнем магнитном поле}
Для полноты изложения в этом разделе обсудим влияние внешней активной среды на 
волновые функции электронов~\cite{KM_Book_2013}, которые являются решением 
уравнения Дирака в присутствии внешнего постоянного однородного магнитного 
поля, направленного вдоль оси $z$:
\begin{equation}\label{eq:Dirac}
	(i\partial_\mu \gamma^\mu -e A_\mu \gamma^\mu - m) \Psi^s_{p,n}(X)=0\, ,
\end{equation}
где  $A^\mu=\left(0,0,xB,0\right)$ -- 
4-вектор потенциала электромагнитного поля в калибровке 
Ландау, $X^\mu=\left(t,x,y,z\right)$. Решением этого уравнения является набор собственных функций любого оператора, который 
коммутирует с гамильтонианом Дирака во внешнем магнитном поле: $H = \gamma_0 
\left( {\bs \gamma} {\bs P} \right) + m \, \gamma_0 -e A_0$, где $\vec{P} 
= -i \vec{\nabla} +e \vec{A}$. Существует несколько представлений решений 
уравнения Дирака, из них можно выделить два наиболее распространенных подхода, 
подробное описание которых имеется в 
работах~\cite{Melrose:1983,Sokolov:1986,Kuznetsov:2003,Bhattacharya:2004,Balantsev:2011,KM_Book_2013}.
 При первом из них, предложенным Джонсоном и Липпманом~\cite{Johnson:1949}, 
решения выбираются как собственные функции оператора обобщенной спиральности, 
$T_0 = \frac{1}{m} (\bs\Sigma \vec{P})$, где $\bs\Sigma = - \gamma_0 
\bs\gamma \gamma_5$  -- трехмерный оператор спина. При этом две верхние 
компоненты биспиноров соответствуют состояниям электрона с проекцией спина на 
направление магнитного поля, равной $1/2$ и   $-1/2$.

Другой подход предложен Соколовым и Терновым~\cite{Sokolov:1968}. Он состоит в выборе волновых функций как собственных функций ковариантного оператора~$\mu_z$, который строится следующим образом: 
\begin{equation}\label{eq:muz}
	\mu_z=m \Sigma_z - i \gamma_0\gamma_5\left[\vec{\Sigma}\times 
	\vec{P}\right]_z\, .
\end{equation}

Его можно получить непосредственно 
из введенного в~\cite{Sokolov:1968} обобщенного оператора спина, 
являющегося тензором третьего ранга, который можно записать в координатном представлении следующим образом:
%
\begin{eqnarray}
	{\rm F}_{\mu \nu \lambda} = - \frac{\ii}{2} \left( P_\lambda \gamma_0 \sigma_{\mu \nu} 
	+ \gamma_0 \sigma_{\mu \nu} P_\lambda \right),
	\label{eq:Fgen}
\end{eqnarray}
%
\noindent где $\sigma_{\mu \nu} = (\gamma_\mu \gamma_\nu - \gamma_\nu \gamma_\mu)/2$, и  
$P_\lambda = \ii \partial_\lambda +e \, A_\lambda = \left( \ii \partial_0 +e \, A_0 \,, 
- \ii {\bs \nabla} +e {\bf A} \right)$ -- оператор обобщенного 4-импульса. 
Заметим, что в работе~\cite{Sokolov:1968} ковариантные билинейные формы были построены из матриц Дирака в обкладках  биспиноров 
$\psi^{\dagger}$ и $\psi$, тогда как в современной литературе (см., например~\cite{Peskin:1995}) билинейные формы строятся из матриц 
Дирака в обкладках биспиноров $\bar\psi$ и~$\psi$. Из пространственных компонент ${\rm F}_{\mu \nu 0}$ 
оператора~(\ref{eq:Fgen}) можно построить следующий векторный оператор:
%
\begin{eqnarray}
	\mu_i = - \frac{1}{2} \, \varepsilon_{ijk} \, {\rm F}_{jk0} \,, 
	\label{eq:mu_i}
\end{eqnarray}
%
где $\varepsilon_{ijk}$ -- тензор Леви-Чивита.  
Построенный таким образом объект~(\ref{eq:mu_i})  имеет смысл 
оператора поляризации~\cite{Sokolov:1968,Melrose:1983}.
Его можно представить в виде:
%
\begin{eqnarray}
	{\bs \mu} = m {\bs \Sigma} + \ii \gamma_0 \gamma_5 [{\bs \Sigma} \times 
	{\bs P}] \, .
	\label{eq:mu_vec}
\end{eqnarray}

В нерелятивистском пределе оператор~(\ref{eq:mu_vec}), 
отнесенный к квадрату массы электрона:  ${\bs \mu}/m^2$,  
переходит в обычный оператор Паули для магнитного момента~\cite{Landau:1989}, 
который имеет явную физическую интерпретацию оператора спина.

Решения уравнения Дирака в представлении Джонсона и Липпмана широко используются в литературе (см., например,~\cite{Canuto:1975,Harding:1991,Suh:1999,Gonthier:2000,Jones:2010,Melrose:2020}). Однако эти функции обладают рядом недостатков, которые проявляются при расчете конкретных характеристик процессов с двумя и более вершинами. Так, лоренц-инвариантностью будет обладать только квадрат модуля амплитуды, просуммированный по всем поляризациям электрона, а не парциальные вклады в него. Более того, как было показано в работах~\cite{Graziani:1993,Gonthier:2014}, в области резонанса использование функций Джонсона и Липпмана приводит к относительной ошибке в расчетах физических величин порядка $O(B / B_{e})$ в древесном приближении и $O[(B / B_{e})^2]$ в следующих порядках разложения, что становится существенным при магнитарных магнитных полях.

С другой стороны, использование функций, предложенных Соколовом и Терновым, правильно описывает сечение процессов вблизи резонанса, а также позволяет найти парциальные вклады в амплитуду каждого поляризационного состояния частиц в отдельности, которые будут иметь лоренц-инвариантную структуру. По этой причине далее в этом разделе приведем подробное их описание. 

Уравнение для собственных функций оператора~(\ref{eq:muz}) имеет следующий вид:
\begin{equation}
	{\mu}_z \Psi^s_{p,n}(X)=s M_n \Psi^s_{p,n}(X)\, ,
\end{equation}
где квантовое число $s=\pm 1$ определяет поляризационные состояния электрона в постоянном однородном магнитном поле.

Как уже упоминалось во Введении, состояния электрона квантуются по энергетическим состояниям, которые называются уровнями Ландау:
\begin{equation}\label{eq:Energy_n}
	E_n = \sqrt{p_z^2+M_n^2}\, ,\,\,\, n=0,1\dots \, .
\end{equation}

Здесь введено обозначение для эффективной массы электрона в магнитном поле 
$M_n=\sqrt{2 \beta n + m^2}$, где $\beta=eB$. Каждое состояние является 
бесконечно вырожденным по $p_z$ и дважды вырожденным по $s$, кроме состояния 
$n=0$, где возможно лишь состояние с $s=-1$. Решения уравнения 
Дирака~(\ref{eq:Dirac}) могут быть представлены следующим образом:
\begin{eqnarray}
\label{eq:psie}
\Psi^s_{p,n}(X) = \frac{e^{-\ii(E_{n} X_0 - p_y X_2 - p_z X_3)}\; U^s_{n} (\xi)}
{\sqrt{4E_{n}M_n (E_{n} + M_n)(M_n + m) L_y L_z}} \, ,  
\end{eqnarray}
где 
\begin{equation}
	\xi(X_1)=\sqrt{\beta}\left(X_1+ \frac{p_y}{\beta}\right)\, .
\end{equation}

Для электрона биспиноры  $U^s_{n} (\xi)$ имеют следующий вид:

\beq
\label{eq:U--}
&&U^{-}_{n} (\xi) = \left ( 
\begin{array}{c}
	-\ii\sqrt{2\beta n} \, p_z V_{n-1} (\xi)\\[2mm]
	(E_n + M_n)(M_n + m) V_n (\xi)\\[2mm]
	-\ii\sqrt{2\beta n} (E_n + M_n) V_{n-1} (\xi)\\[2mm]
	-p_z (M_n + m) V_n (\xi)
\end{array}
\right )  ,   
%
\\ [3mm]
\label{eq:U+-}
&&U^{+}_{n} (\xi) = \left ( 
\begin{array}{c}
	(E_n + M_n) (M_n + m) V_{n-1} (\xi)\\[2mm]
	-\ii\sqrt{2\beta n} \, p_z V_n (\xi)\\[2mm]
	p_z (M_n + m) V_{n-1} (\xi)\\[2mm]
	\ii \sqrt{2 \beta n} (E_n + M_n) V_n (\xi)
\end{array}
\right )\! .
\eeq

Для позитрона биспиноры  $U^s_{n} (\xi)$ имеют следующий вид:
%
\beq
\label{eq:U-+}
&&U^{-}_{n} (\xi) = \left ( 
\begin{array}{c}
	\ii\sqrt{2\beta n} \, p_z V_{n} (\xi)\\[2mm]
	(E_n + M_n)(M_n + m) V_{n-1} (\xi)\\[2mm]
	\ii\sqrt{2\beta n} (E_n + M_n) V_{n} (\xi)\\[2mm]
	-p_z (M_n + m) V_{n-1} (\xi)
\end{array}
\right ) \!\! ,   
%
\\ [3mm]
\label{eq:U++}
&&U^{+}_{n} (\xi) = \left ( 
\begin{array}{c}
	(E_n + M_n) (M_n + m) V_{n} (\xi)\\[2mm]
	\ii\sqrt{2\beta n} \, p_z V_{n-1} (\xi)\\[2mm]
	p_z (M_n + m) V_{n} (\xi)\\[2mm]
	-\ii \sqrt{2 \beta n} (E_n + M_n) V_{n-1} (\xi)
\end{array}
\right ) , 
\eeq
%
$V_n(\xi)$ -- нормированные функции гармонического осциллятора, которые 
следующим образом выражаются через полиномы Эрмита $H_n(\xi)$ \cite{Gradstein:1963}:
%
\begin{eqnarray}
\label{eq:V_n}
V_n (\xi) = \frac{\beta^{1/4}\eee^{-\xi^2/2}}{\sqrt{2^n n! \sqrt{\pi}}} \, H_n(\xi)\, .
\end{eqnarray}
