\subsection{Пропагатор фермиона во внешнем магнитном поле}
При рассмотрении любого взаимодействия в квантовой теории поля предполагается, что между начальными и конечными состояниями существует обмен виртуальными частицами. "Виртуальность" частицы означает, что ее энергия и импульс не связаны релятивистским соотношением~(\ref{eq:Energy_n}). Промежуточное состояние в формализме собственного времени Фока~\cite{Schwinger:1951} описывается уравнением Дирака с $\delta$-функцией в правой части:
%
\begin{equation}\label{eq:Green}
(\ii\partial_\mu \gamma^\mu -e A_\mu \gamma^\mu - m_e) S(X,X')=\delta\left(X-X'\right)\, .
\end{equation}
%

Его решение $S(X,X')$ называется пропагатором. В этом разделе мы опишем представление пропагаторов фермионов во внешнем магнитном поле с учетом радиационных поправок к массовому оператору и покажем, как может реализовываться резонанс в квантовых процессах, содержащих фермионы в промежуточном состоянии.

Решения уравнения~(\ref{eq:Green}) имеют достаточно громоздкий вид. Поэтому удобно воспользоваться различными приближениями, \textcolor{red}{например}, пропагатор удобно рассматривать в виде разложения по уровням Ландау:
\begin{equation}
	S(X,X') = \sum_{n=0}^{\infty}\sum_{s=\pm 1} S_n^s (X,X')\, .
\end{equation}

С другой стороны, для построения пропагатора, удовлетворяющего уравнению~(\ref{eq:Green}), можно воспользоваться полевыми операторами:
%
\begin{eqnarray}
\Psi (X) = \sum\limits_{n,p_y,p_z,s} ( a^{s}_{n,p} \Psi_{n,p,+}^s (X) + b^{\dagger s}_{n,p} \Psi_{n,p,-}^s (X) )\,,
\end{eqnarray}
\noindent где $a$ -- оператор уничтожения фермиона, $b^{\dagger}$ -- оператор рождения фермиона, $\Psi_{+}$ и $\Psi_{-}$ соответствуют решениям уравнения Дирака с положительной и отрицательной энергией соответственно. Следует упомянуть, что влиянием плазмы на виртуальные состояния пренебрегаем[\textcolor{red}{Эминов, Вшивцев}]. Стандартным образом пропагатор вычисляется как разность хронологически упорядоченного и нормально упорядоченного произведения полевых \textcolor{red}{операторов~\cite{KM_Book_2013}:}
%
\begin{equation}\label{eq:SXX}
S(X,X') = T(\Psi(X)\bar{\Psi}(X')) - {\cal N}(\Psi(X)\bar{\Psi}(X'))\, .
\end{equation}
%

Подставляя в~(\ref{eq:SXX}) точные решения уравнения Дирака~(\ref{eq:psie}) и вводя для удобства новое обозначение:
\begin{eqnarray}
\phi^{s}_{p, n} (X_1) =   \frac{U^s_{n} [\xi(X_1)]}
{\sqrt{2 M_n (E_{n} + M_n)(M_n + m_e)}} \, ,
\label{eq:phi_psi}
\end{eqnarray}
%
\noindent где $U^s_{n}$ определяется формулой~(\ref{eq:U^s}), можно представить вклад в разложение пропагатора от уровня Ландау $n$ и поляризационного состояния $s$ следующим образом:
\begin{eqnarray}
\label{eq:propagator}
S^{s}_n (X, X^{\,\prime}) && =  \int \frac{\dd p_0 \dd p_y \dd p_z}{(2\pi)^3} \times
\\[3mm]
\nonumber
&& \times \frac{\eee^{- \, \ii \,  p_0 \,(X_0 - X^{\,\prime}_0) + 
		\ii p_y \,(X_2 - X^{\,\prime}_2) +  \ii \,  p_z \,(X_3 - X^{\,\prime}_3)}}
{p_0^2 - p_z^2 - M_n^2 - {\cal R}^{s}_\Sigma (p) + \ii\, {\cal I}^{s}_\Sigma (p)}
\, \phi^{s}_{p, n} (X_1) \bar \phi^{s}_{p, n} (X'_1) \, ,
\end{eqnarray}
где ${\cal R}^{s}_\Sigma (p)$ и ${\cal I}^{s}_\Sigma (p)$ -- реальная и мнимая части массового оператора фермиона.  Для их получения требуется вычислить радиационные поправки к массе фермиона в замагниченной плазме. Реальная часть массового оператора ${\cal R}^{s}_\Sigma (p)$ определяет изменение закона дисперсии фермиона в присутствии замагниченной плазмы. В слабых магнитных полях, где выполняется $B\ll B_e$, но среду ещё можно рассматривать как поледоминирующую, она определяется 
отношением~\cite{Ritus1969}:
%
\begin{eqnarray}
\label{eq:Re1}
\Re^{s}_\Sigma (p) = \frac{4\alpha m_e}{3\pi} \varkappa^2 \left [ \ln \varkappa^{-1} + C + \frac{1}{2} \ln 3 - \frac{33}{16} \right],\quad \varkappa\ll 1,
\end{eqnarray}
\noindent где $C = 0.577...$ - постоянная Эйлера, динамический параметр $\varkappa$ вводится следующим образом:
%
\begin{eqnarray}
\varkappa = \frac{1}{m_e B_e} [-(F_{\mu\nu} p_{\nu})^2]^{1/2}.
\end{eqnarray}
%

Для случая сильного магнитного поля, $B\gtrsim B_e$, без учета плазмы лидирующий вклад в  сдвиг массы фермиона, находящегося на основном уровне Ландау, описывается квадратом логарифмической функции~\cite{Jancovici:1969}:
\begin{eqnarray}
\label{eq:Re2}
\Re^{s}_\Sigma (p) = \frac{\alpha}{4\pi} m_e \ln^2 (2\beta/m_e^2).
\end{eqnarray}
%

Из (\ref{eq:Re1}) и (\ref{eq:Re2}) следует, что даже для достаточно больших значений магнитного поля вплоть до $10^{16}$ Гс эта поправка к массе фермиона имеет величину порядка постоянной тонкой структуры $\alpha$~\cite{Kuznetsov:2003,Sokolov:1986} и является несущественной.
 
Резонанс на виртуальном фермионе будет наблюдаться, когда в знаменателе пропагатора~(\ref{eq:propagator}) реальная часть обращается в ноль. Тогда виртуальная частица становится реальной, то есть приобретает определенный закон дисперси~(\ref{eq:Energy_n}). Анализ кинематики показывает, что это возможно только тогда, когда виртуальный фермион занимает один из высших уровней Ландау, $n > 0$. Частица при этом является нестабильной, и время ее жизни, в нерезонансной области предполагающееся бесконечно большим, определяется мнимой частью массового оператора,  ${\cal I}^{s}_\Sigma (p)$, учет которой становится необходимым. Она может быть получена с помощью оптической теоремы и представлена в следующем виде~\cite{Weldon:1982, Zhukovski:1994}:
%
\begin{eqnarray}
\Im^{s}_\Sigma (p) = - \frac{1}{2}\, p_0 \; \Gamma_n^{s} \, ,
\label{eq:I_Sigma}
\end{eqnarray}
\noindent где $\Gamma_n^{s}$ -- полная ширина изменения состояния фермиона, 
находящегося 
в поляризационном состоянии $s$ и занимающего  $n$-й уровень Ландау.  Введенный 
таким образом пропагатор с учетом конечной ширины изменения состояния фермиона  
позволяет корректно рассчитывать сечения квантовых процессов в резонансной 
области.