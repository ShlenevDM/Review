\section{Введение}
Резонансные явления в нашей жизни встречаются повсеместно. Большинство этих явлений так или иначе связано с колебательными процессами. Так, резонанс в колебательном контуре даeт возможность получать или передавать информацию 
на определенной частоте с минимально затраченной энергией. При этом, в классическом подходе резонанс -- это резкое возрастание амплитуды колебаний при приближении частоты внешнего воздействия к собственной частоте системы. С другой стороны, в квантовом подходе в силу существования дискретных уровней энергии резонанс означает резкое увеличение вероятностей процессов, связанных с переходами между этими уровнями, таких как, например, поглощение фотона с энергией, равной разнице энергий между двумя состояниями электрона. Одним из условий, при которых учет квантовых эффектов при движении частиц становится необходимым, является присутствие сильных магнитных полей, чья индукция приближается к характерному значению, называемому критическим, $B_e = m_e^2 / e \simeq 4.41 \times 10^{13}$ Гс \footnote{В работе используется естественная система единиц: $\hbar = c = k = 1$, $m_e$ -- масса электрона, $e > 0$ --  элементарный заряд.}. В природе такими экстремально большими магнитными полями, согласно современным моделям~\cite{Duncan:1995,Thompson:1996,Thompson:2002}, обладают разновидности нейтронных звезд, называемые радиопульсарами (с магнитными полями порядка $10^{12}$ Гс) и магнитарами (до $10^{15}$ Гс)~\cite{McGill:Cite}.

Кроме сильных магнитных полей в магнитосфере как радиопульсаров, так и магнитаров присутствует относительно горячая и плотная электрон-позитронная плазма~\cite{Duncan:1995}. Магнитное поле и плазма составляют 
две компоненты внешней активной среды, присутствие которой значительно изменяет характеристики протекающих в ней микропроцессов. Во-первых, активная среда может изменять закон дисперсии 
находящихся в ней частиц, что приводит к изменению кинематики процессов и вследствие чего могут открываться  каналы реакций, которые запрещены или сильно подавлены в вакууме. Во-вторых, активная среда 
влияет на амплитуды процессов, в результате чего они могут приобретать резонансный характер. Именно эта составляющая влияния внешней активной среды рассматривается в данном обзоре. 
Вследствие резонанса вклад микропроцессов в макроскопические характеристики астрофизических процессов, такие как светимость и скорость изменения количества частиц, может 
многократно увеличиваться.

\textcolor{red}{Радиопульсары -- это быстровращающиеся одиночные нейтронные звезды, которые демонстрируют периодические пульсации в радиочастотном диапазоне электромагнитного спектра со стабильным периодом. Согласно современным моделям~\cite{Gunn:1969,Pacini:1970} основным механизмом потери энергии вращения для радиопульсаров с периодами вращения $P<2$ c является магнито-дипольное излучение. Исходя из этого, была получена оценка магнитных полей на поверхности радиопульсаров~\cite{Kim:2023}: $B\sim 10^{10}-10^{14}$ Гс для среднепериодических ($0.1\text{ с}<P<2$~с) пульсаров и $B\sim10^8-10^{14}$~Гс для короткопериодических ($P<0.1$ с). Около 110 из 1468 представленных в работе~\cite{Kim:2023} объектов обладают магнитными полями порядка критического значения, достигая для одного из них максимального значения напряженности магнитного поля $B\simeq7.56 \times 10^{14}$~Гс. Вблизи полярных шапок радиопульсаров сильное магнитное поле отклоняет и ускоряет заряженные частицы, что приводит к генерации электромагнитного излучения.    Несмотря на достаточно долгие наблюдения радиопульсаров, их радиоизлучение остается загадочным явлением, один из возможных механизмов которого исследовался, например, в работе~\cite{Philippov_2020}.}
	
\textcolor{red}{Другими объектами с полями, масштаба критического значения, являются рентгеновские пульсары -- сильно замагниченные нейтронные звезды, находящиеся в тесной двойной системе с обычной звездой. Достаточно сильное магнитное поле $B\gtrsim10^{12}$ Гс в этом случае существенно влияет на путь аккреционного потока. Вещество в виде плазмы, аккрецирующееся на нейтронную звезду, следует линиям магнитного поля и сосредотачивается в относительно малых областях на поверхности звезды, близких к магнитным полюсам (т.н.~полярным шапкам). В данной горячей области $T\sim10^9-10^{10}$ К кинетическая энергия выделяется преимущественно в виде рентгеновских лучей. В спектре этих объектов присутствую циклотронные особенности, которые впервые были открыты в 1977 году~\cite{Trumper:1977}, в области энергий от $10$ кэВ до $100$ кэВ приблизительно. Наличие данных циклотронных линий позволило прямо измерить значения магнитного поля~\cite{Staubert:2019}: $B\sim10^{12}$ Гс.}
 
%Обзор состава поверхностного слоя нейтронной звезды в сильном магнитном поле представлен в работе~\cite{DongLai:2001}.

\textcolor{red}{Наконец, существуют так называемые магнитары -- отдельный класс изолированных нейтронных звезд, значение магнитных полей которых достигает $10^{14} - 10^{15}$ Гс~\cite{Mitrofanov:1982,Duncan:1992,Thompson:1995,Thompson:1996,DongLai:2001,Lyutikov:2002}. Исторически сложилось, что магнитары подразделяют на источники мягких повторяющихся гамма-всплесков (SGR -- Soft Gamma-Repeater) и аномальные рентгеновские пульсары (AXP -- Anomalous \linebreak X-ray pulsar)~\cite{Kouveliotou:1998ze,Kouveliotou:1998fd,Gavriil:2002mc,Ibrahim:2002zw,Ibrahim:2002zy,Olausen:2014,vanParad:1995}. Первыми были открыты SGR при наблюдении повторяющихся интенсивных вспышек в жестком рентгеновском и мягком диапазоне~\cite{Mazets:1979}. В свою очередь, AXP были впервые замечены в области мягкого гамма-излучения ($<10$~кэВ) и изначально предполагалось, что они принадлежат двойным аккрецирующим системам~\cite{Mereghetti:1995}. Для магнитаров характерны пульсации с достаточно большим периодом от $2$ до $12$ секунд, а также рентгеновским излучением в области $0.5 - 10$~кэВ и $20 - 100$~кэВ. При этом температура поверхности имеет порядок $T\sim 10^6$ К~\cite{Yakovlev:2004}. Помимо их основных характеристик, в магнитарах также проявляется вспышечная активность. Как для SGR, так и для AXP характерны короткие вспышки продолжительностью от $0.1$  до $1$ секунды, которые могут наблюдаться также в области низких энергий ($\sim 10$~кэВ), однако пиковое значение находится в области высоких энергий (до $100$~кэВ)~\cite{Younes:2021}. Наиболее редкими явлениями, которые наблюдались только в SGR, являются гигантские вспышки~\cite{Mazets:1979,Hurley:1999,Hurley:1999b,Hurley:2005}. Данные явления наблюдались у источников, магнитные поля которых являются одними из самых больших (от $10^{14}$ Гс до $10^{15}$ Гс).  В результате гигантских вспышек из SGR высвобождается огромное количество энергии, что приводит к наблюдаемому излучению в области очень высоких энергий до $2$ МэВ. Некоторые спектральные модели гигантских вспышек предполагают температуры в области \mbox{$2-3\times 10^9$} К, однако в пиковом значении могут достигать и выше\linebreak $T\sim 10^{10}$ К~\cite{Hurley:1999}. Наиболее подробный обзор наблюдаемых данных и физических процессов, происходящих в магнитарах, можно найти, например, в работе~\cite{Kaspi:2017}. }

Как известно, в сильном магнитном поле поперечная составляющая импульса электрона квантуется. В таком случае энергия электрона определяется так называемым уровнем Ландау $n$ и проекцией импульса вдоль магнитного поля $p_z$ и в пренебрежении аномальным магнитным моментом электрона выражается следующим образом~\cite{Sokolov:1968}: 
\begin{equation}
E_n = \sqrt{m_e^2+p_z^2+2 e B n}.
\end{equation}
%
При этом интерпретация состояния с $n=0$ (т.н. основной уровень Ландау) имеет различную трактовку. Так в классическом представлении такому состоянию будет соответствовать движение электрона вдоль силовой линии магнитного поля [\textcolor{red}{Блохинцев?}]. В квантовом подходе состоянию с $n=0$, например, соответствует ненаблюдаемость поперечного движения электрона по отношению к направлению магнитного поля~\cite{KM_Book_2013}. В частности для полей $B\gg B_e$ электроны будут преимущественно занимать основной уровень Ландау. 

Учет влияния макроскопических коллективных состояний электронов (позитронов), например, плазмы температуры $T$ и химическом потенциале $\mu$ приводит к модификации этого условия \cite{Borisov:1997}: $eB \gg \mu^2, T^2, E^2$, где $E$ -- энергия электронов среды. Более строгое соотношение между параметрами, при выполнении которого можно говорить о пределе сильного поля, получается из условия того, что в этом случае плотность энергии магнитного поля во много раз превосходит плотность энергии электрон-позитронного газа~\cite{KuzMih:2000}: 
%
\beq
\label{bigB}
\frac{B^2}{8\pi} \gg \frac{\pi^2 (n_{e^{-}} - n_{e^{+}})^2}{e B} + \frac{eBT^2}{12}\,,
\eeq 
%
\noindent где $n_{e^{-}}$ и $n_{e^{+}}$ -- концентрации электронов и позитронов плазмы. Такие условия могут, в частности, реализовываться в моделях вспышечной активности источников мягких повторяющихся гамма-всплесков 
(SGR)~\cite{Duncan:1995, Bisnovatyi:1979}, которые, как показывают недавние наблюдения, можно отождествить с магнитарами~\cite{Kouveliotou:1998ze,Kouveliotou:1998fd,Gavriil:2002mc,Ibrahim:2002zw,Ibrahim:2002zy,Olausen:2014}.

С другой стороны, даже в магнитарных полях условие~(\ref{bigB}), при котором магнитное поле является доминирующим параметром, перестает выполняться при высоких значениях плотности плазмы $\rho \gtrsim 10^8$ г/см$^3$. Такая плотность может достигаться в  границе между внешней и внутренней корой магнитара. В этом случае электроны (реальные и виртуальные), начинают эффективно заполнять следующие уровни Ландау, что приводит к возможности возникновения резонансов в реакциях, где электроны в промежуточном состоянии, могут приобретать \textcolor{red}{резонансный} характер.
 В результате резонанса виртуальные частицы выходят на массовую поверхность, т.е. становятся реальными с определенным законом дисперсии. 
Однако в этом состоянии они являются нестабильными и могут распадаться за время, обратно пропорциональное вероятности их перехода на низшие уровни Ландау. Эффективность реакций при этом заметно увеличивается, что 
может иметь наблюдаемые астрофизические следствия [\textcolor{red}{какие?+ссылки}].

\textcolor{red}{Одним из ярких представителей реакций, эффективность которых значительно увеличивается на резонансных энергиях фотона, является процесс комптоновского  рассеяния фотонов на электронах (позитронах) $\gamma e \to \gamma e$ 
замагниченной среды, которое играет ключевую роль в формировании спектров нейтронных звезд~\cite{Miller:1995,Bulik:1997,Suleimanov:2007it,Nobili:2008,Taverna:2014}.
Под влиянием сильного магнитного поля становятся возможны резонансы, связанные с переходами электронов между уровнями Ландау. В~результате сечение комптоновского рассеяния на резонансных частотах, которые называются циклотронными резонансами, становится много больше томсоновского значения $\sigma_T$. Таким образом комптоновский процесс приводит к появлению циклотронных особенностей замагниченных нейтронных звезд, а также влияет на взаимодействие теплового излучения аккрецирующего вещества и поверхностного излучения в SGR. Проявление циклотронного резонанса на частоте $\omega_B=eB/mc$ позволило дать оценку величине магнитного поля нейтронных звезд~\cite{Mitrofanov:1982}.  На данный момент известно около 36 пульсаров с циклотронными особенностями в их рентгеновском спектре~\cite{Staubert:2019}. Для магнитаров резонансный комптоновский процесс является частью механизма генерации низкоэнергетического спектра~\cite{Lyutikov:2002,Rea:2008}. В частности, в работе~\cite{Rea:2008} модель резонансного комптоновского рассеяния используется для моделирования спектра, который с достаточно хорошей точностью удовлетворяет наблюдаемым спектрам магнитаров в мягком рентгеновском диапазоне. Таким образом исследование комптоновского процесса в экстремальных условиях является интересной научной задачей.}

Резонанс на фотоне наблюдается аналогичным образом: во внешней активной среде его поляризационный оператор приобретает реальную часть, которую можно рассматривать как эффективную массу фотона. В кинематической 
области, в которой квадрат 4-импульса виртуального фотона равен реальной части его поляризационного оператора, виртуальный фотон становится реальным и нестабильным [\textcolor{red}{ссылка}].

Настоящая статья организована следующим образом. В разделе 2 описывается влияние внешнего магнитного поля на движение электронов, обсуждаются различные методы представления решения уравнения Дирака во внешнем магнитном поле и получается выражение для пропагатора. В разделе 3 рассматривается распространение радиации в магнитном поле и представлен поляризационный оператор фотона. Раздел 4 посвящен различным двухвершинным процессам, в которых может 
реализовываться резонанс на виртуальном фермионе и/или фотоне. В разделе 5 описываются сингулярности в фазовых объемах одновершинных процессов и методы их устранения.