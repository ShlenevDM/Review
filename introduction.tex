Нейтронные звёзды, обладая набором экстремальных характеристик, являются природными физическими лабораториями и одними из самых интересных объектов, известных в науке. 
Особое внимание учёных привлекают радиопульсары и магнитары, обладающие магнитными полями колоссальной напряжённости, к которой очень сложно приблизиться в
земных условиях. У магнитного поля имеется характерное значение, называемое критическим, $B_e = m^2 / e \simeq 4.41 \times 10^{13}$ Гс \footnote{В работе используется естественная система единиц: 
$\hbar = c = k = 1$, $m$ -- масса электрона, $e > 0$ --  элементарный заряд.}, при приближении к которому становится необходимым учитывать квантовые эффекты при движении в нём частиц. 
В радиопульсарах с магнитными полями порядка $10^{12}$ Гс и магнитарах -- до $10^{16}$ Гс~\cite{Duncan:1995,Thompson:1996,Lyutikov:2002} такие условия выполняются.

Кроме сильных магнитных полей, в магнитосфере как радиопульсаров, так и магнитаров, присутствует достаточно плотная электрон-позитронная плазма. Магнитное поле и плазма составляют 
две компоненты внешней активной среды, присутствие которой значительно изменяет характеристики протекающих в ней микропроцессов. Во-первых, активная среда может изменять закон дисперсии 
находящихся в ней частиц, что приводит к изменению кинематики процессов и вследствие чего могут открываться реакции и каналы реакций, которые запрещены в вакууме. Во-вторых, активная среда 
влияет на амплитуды процессов, в результате чего они могут приобретать резонансный характер. Именно эта составляющая влияния внешней активной среды рассматривается в данном обзоре. 
Вследствие резонанса вклад микропроцессов в макроскопические характеристики астрофизических процессов, такие как светимость и скорость изменения количества частиц, может 
многократно увеличиваться.

В сильном магнитном поле поперечная составляющая импульса фермиона квантуется. В таком случае энергия фермиона определяется так называемым уровнем Ландау $n$ и проекцией импульса вдоль магнитного поля $p_z$:
\begin{equation}
E_n = \sqrt{1+p_z^2+2\beta n},
\end{equation}
%
где введено обозначение $\beta=|e_f|B$, $e_f$ - заряд фермиона. Состояние с $n = 0$, в котором фермион движется вдоль силовой линии магнитного поля, называется основным уровнем Ландау. 

Можно выделить несколько ситуаций в иерархии параметров среды: магнитного поля, температуры $T$, химического потенциала $\mu$ и энергии фермионов и фотонов, участвующих в реакциях.
Предел сильного поля, когда фермионы будут занимать основной уровень Ландау, осуществляется при выполнении условия~\cite{RCh:2008}:
%
\beq
\label{bigB}
\frac{B^2}{8\pi} \gg \frac{\pi^2 (n_{e^{-}} - n_{e^{+}})^2}{e B} + \frac{eBT^2}{12}\,,
\eeq 
%
\noindent где $n_{e^{-}}$ и $n_{e^{+}}$ -- концентрации электронов и позитронов плазмы.

При значениях плотности плазмы $\rho \geqslant 10^8$ г/см$^3$ условие~\ref{bigB} перестаёт выполняться и начинают возбуждаться 
высшие уровни Ландау виртуальных фермионов, в результате чего они становятся нестабильными. 