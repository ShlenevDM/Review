\section{Введение}
Резонансные явления в нашей жизни встречаются повсеместно. Большинство этих явлений так или иначе связано с колебательными процессами. Так, резонанс в колебательном контуре даeт возможность получать или передавать информацию 
на определенной частоте с минимально затраченной энергией. В классическом подходе резонанс -- это резкое возрастание амплитуды колебаний при приближении частоты внешнего воздействия к собственной частоте системы. В квантовом подходе в силу существования дискретных уровней энергии резонанс означает резкое увеличение вероятности процессов, связанных с переходами между этими уровнями, таких как, например, поглощение фотона с энергией, равной разнице энергий между двумя состояниями. Одним из условий, при которых учет квантовых эффектов при движении частиц становится необходимым, является присутствие сильных магнитных полей, чья индукция приближается к характерному значению, называемому критическим, $B_e = m_e^2 / e \simeq 4.41 \times 10^{13}$ Гс \footnote{В работе используется естественная система единиц: $\hbar = c = k = 1$, $m_e$ -- масса электрона, $e > 0$ --  элементарный заряд.}. В природе такими экстремально большими магнитными полями, согласно современным моделям~\cite{Duncan:1995,Thompson:1996,Thompson:2002}, обладают разновидности нейтронных звезд, называемые радиопульсарами (с магнитными полями порядка $10^{12}$ Гс) и магнитарами (до $10^{15}$ Гс).

Кроме сильных магнитных полей в магнитосфере как радиопульсаров, так и магнитаров присутствует относительно горячая и плотная электрон-позитронная плазма~\cite{Duncan:1995}. Магнитное поле и плазма составляют 
две компоненты внешней активной среды, присутствие которой значительно изменяет характеристики протекающих в ней микропроцессов. Во-первых, активная среда может изменять закон дисперсии 
находящихся в ней частиц, что приводит к изменению кинематики процессов и вследствие чего могут открываться  каналы реакций, которые запрещены или сильно подавлены в вакууме. Во-вторых, активная среда 
влияет на амплитуды процессов, в результате чего они могут приобретать резонансный характер. Именно эта составляющая влияния внешней активной среды рассматривается в данном обзоре. 
Вследствие резонанса вклад микропроцессов в макроскопические характеристики астрофизических процессов, такие как светимость и скорость изменения количества частиц, может 
многократно увеличиваться.

Как известно, в сильном магнитном поле поперечная составляющая импульса электрона квантуется. В таком случае энергия электрона определяется так называемым уровнем Ландау $n$ и проекцией импульса вдоль магнитного поля $p_z$ и в пренебрежении аномальным магнитным моментом электрона выражается следующим образом~\cite{Sokolov:1968}: 
\begin{equation}
E_n = \sqrt{m_e^2+p_z^2+2 e B n}.
\end{equation}
%
Состояние с $n = 0$, в котором электрон движется вдоль силовой линии магнитного поля, называется основным уровнем Ландау.

В случае, если значение индукции магнитного поля существенно превышает остальные параметры среды: $eB \gg \mu^2, T^2, E^2$, где $\mu$ -- химический потенциал электронов, $E$ -- энергия электронов среды, $T$ -- температура плазмы, расчет макроскопических характеристик значительно упрощается. Более строгое соотношение между параметрами, при выполнении которого можно говорить о пределе сильного поля, получается из условия того, что в этом случае плотность энергии магнитного поля во много раз превосходит плотность энергии электрон-позитронного газа~\cite{KuzMih:2000}: 
%
\beq
\label{bigB}
\frac{B^2}{8\pi} \gg \frac{\pi^2 (n_{e^{-}} - n_{e^{+}})^2}{e B} + \frac{eBT^2}{12}\,,
\eeq 
%
\noindent где $n_{e^{-}}$ и $n_{e^{+}}$ -- концентрации электронов и позитронов плазмы. Такие условия могут, в частности, реализовываться в моделях вспышечной активности источников мягких повторяющихся гамма-всплесков 
(SGR)~\cite{Duncan:1995, Bisnovatyi:1979}, которые, как показывают недавние наблюдения, можно отождествить с магнитарами~\cite{Kouveliotou:1998ze,Kouveliotou:1998fd,Gavriil:2002mc,Ibrahim:2002zw,Ibrahim:2002zy,Olausen:2014}.

С другой стороны, даже в магнитарных полях условие~(\ref{bigB}), при котором магнитное поле является доминирующим параметром, перестает выполняться при высоких значениях плотности плазмы $\rho \gtrsim 10^8$ г/см$^3$. Такая плотность может достигаться в  границе между внешней и внутренней корой магнитара. В результате реакции, в которых электроны (позитроны) находятся в промежуточном состоянии, могут приобретать резонансный характер.
Это происходит вследствие того, что начинают возбуждаться высшие уровни Ландау виртуальных электронов. В результате резонанса они становятся реальными с определенным законом дисперсии, то есть будут находиться на массовой поверхности. 
Однако в этом состоянии они являются нестабильными и могут распадаться за время, обратно пропорциональное вероятности их перехода на низшие уровни Ландау. Эффективность реакций при этом заметно увеличивается, что 
может иметь наблюдаемые астрофизические следствия.

Резонанс на фотоне наблюдается аналогичным образом: в активной среде его поляризационный оператор приобретает реальную часть, которую можно рассматривать как эффективную массу фотона. В кинематической 
области, в которой квадрат 4-импульса виртуального фотона равен реальной части его поляризационного оператора, виртуальный фотон становится реальным и нестабильным.

Настоящая статья организована следующим образом. В разделе 2 описывается влияние внешнего магнитного поля на движение фермионов, обсуждаются различные методы представления решения уравнения Дирака во внешнем магнитном поле и получается выражение для пропагатора. В разделе 3 рассматривается распространение радиации в магнитном поле и представлен поляризационный оператор фотона. Раздел 4 посвящен различным двухвершинным процессам, в которых может 
реализовываться резонанс на виртуальном фермионе и/или фотоне. В разделе 5 описываются сингулярности в фазовых объемах одновершинных процессов и методы их устранения.