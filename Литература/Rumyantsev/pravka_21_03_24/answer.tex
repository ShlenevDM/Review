\documentclass[12pt]{article}

\usepackage{graphics}

\usepackage{amssymb,cite}

%\usepackage[cp1251]{inputenc}
\usepackage[cp866]{inputenc}

\usepackage[english, russian]{babel}
\usepackage{amsbsy}

		\def\mprp{\mbox{\tiny $\bot$}}
		\def\mprl{\mbox{\tiny $\|$}}
\newcommand{\bs}{\boldsymbol} 

\pagestyle{empty}

\begin{document}
\large
\begin{center}

           Ответ на замечания рецензента к статье Д. А. Румянцева, Т.А. Пухова и М. В. Чистякова
      ''Влияние внешней активной среды на процесс двойного комптоновского рассеяния`` 
\end{center}

\centerline{Глубокоуважаемая редакция}

\vspace{5mm}

    Мы доработали статью с учетом замечаний рецензента. 

 
   1. В рецензии предложено пояснить различие в размерностях базисных векторов $b_{\mu}^{(i)}$. 

   Действительно, сами по себе базисные векторы $b_{\mu}^{(i)}$  и $r_{\mu}^{(i)}$ имеют разные 
   размерности, но этот факт не будет влиять на размерность компонент поляризационного оператора, 
   поскольку последний раскладывается уже по нормированным векторам (см. формулу (2)). Кроме того, 
   в исправленной версии статьи показано, что  в силу ненормированности собственных векторов 
   $r_{\mu}^{(1,2)}$  поляризационного оператора в зарядово симметричной плазме для описания 
   поляризационных состояний фотонов в такой плазме удобно ввести нормированные векторы (18), 
   которые затем сравнивать с поляризационными состояниями фотона в чистом магнитном поле.


   2. В формуле (1) величина $(\Lambda q)_\mu$ заменена на эквивалентное представление 
   $(\varphi \varphi q)_\mu$, определяемое в тексте статьи.

   
   3. В рецензии предложено  пояснить, при каких особенностях электрон-позитронной плазмы имеет место 
   отличие или согласие полученных в статье результатов, описывающих поляризацию и дисперсионные 
   свойства нормальных мод с результатами работы Potekhin et al., ApJ 612, 1034 (2004) и других работ.

   
   В доработанном варианте статьи указано, что введенные собственные векторы (18) согласуются в пределе 
   $\omega \ll m$ с точностью до $O(1/\beta)$ и $O(\alpha^2)$ с результатами работы  
   Potekhin et al., ApJ 612, 1034 (2004). А именно: в  этой  работе, в указанном пределе параметр $g \to 0$, 
   так, что параметр $\beta \gg 1$. В этом случае вектор ${\bf e}^{(1)}$  для X-моды связан с введенным 
   нами вектором $\varepsilon^{(1)}_\alpha$ в трехмерных обозначениях следующим образом:
   ${\bs \varepsilon}^{(1)} = - {\bf e}^{(1)}$ (вектор импульса фотона, ${\bf k}$ лежит в плоскости $xz$).
   Для того, чтобы получить пространственную часть вектора $\varepsilon^{(2)}_\alpha$ необходимо совершить 
   над ним калибровочное преобразование, так, что 

   $$\varepsilon^{(\prime 2)}_\alpha = \frac{\varepsilon^{(2)}_\alpha + q_\alpha \varkappa}
   {\sqrt{q^2 \varkappa^2 - 1}} \, ,$$
 
   где $\varkappa = k_z/(\omega \sqrt{q_{\mprl}^2})$. 
   
   Для перехода к системе координат, в которой вектор ${\bf k}$ направлен вдоль оси $z$, 
   нужно полученный трехмерный вектор ${\bs \varepsilon}^{(2)}$ отразить относительно оси $z$ и 
   повернуть на угол $\pi - \theta$. 

   С учетом уравнения дисперсии (22), получим для O-моды  ${\bs \varepsilon}^{(2)} = - {\bf e}^{(2)}$. 
   Следует отметить, что полученное равенство справедливо  с точностью до слагаемого $\alpha/(2\pi) 1.272$, 
   которое отсутствует при вычислении поляризационного оператора фотона в сильном магнитном поле в 
   однопетлевом приближении (см., например, Шабад, 1988).   
      
   

    
   4. В рецензии предложено  написать формулу (31) в явном виде, используя 
   параметры $n_e$ и $T$, как это сделано в формуле (30).

   В доработанный вариант статьи добавлены соответствующие выкладки.    

%\end{enumerate}   
\vspace{5mm}

   С уважением, Д. А. Румянцев,Т.А. Пухов и М. В. Чистяков. 

\end{document}
