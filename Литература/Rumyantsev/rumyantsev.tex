%
% ****** maiksamp.tex 29.11.2001 ******
%
\documentclass[
aps,%
12pt,%
final,%
notitlepage,%
oneside,%
onecolumn,%
nobibnotes,%
nofootinbib,% 
superscriptaddress,%
noshowpacs,%
centertags]%
{revtex4}

\begin{document}
%\selectlanguage{russian}
\selectlanguage{english}

\title{THE INFLUENCE OF EXTERNAL ACTIVE MEDIUM ON THE PROCESS OF DOUBLE COMPTON SCATTERING
}% Разбиение на строки осуществляется командой \\

\author{\firstname{D.~A.}~\surname{Rumyantsev}}
% Здесь разбиение на строки осуществляется автоматически или командой \\
\email{rda@uniyar.ac.ru}
\affiliation{%
P.G. Demidov Yaroslavl State University, 
150003, Russian Federation, Yaroslavl, Sovetskaya Str. 14
}%
%\affiliation{%
%P.G. Demidov Yaroslavl State University
%}%
%\author{\firstname{A.~A.}~\surname{Yarkov}}
%\email{a12l@mail.ru}
%\affiliation{%
%P.G. Demidov Yaroslavl State University, 
%150003, Russian Federation, Yaroslavl, Sovetskaya Str. 14
%}%
%\affiliation{%
%P.G. Demidov Yaroslavl State University
%}%

\author{\firstname{M.~V.}~\surname{Chistyakov}}
\email{mch@uniyar.ac.ru}
\affiliation{%
P.G. Demidov Yaroslavl State University, 
150003, Russian Federation, Yaroslavl, Sovetskaya Str. 14
}%
%\noaffiliation % если у автора место работы не указывается
%\affiliation{%
%P.G. Demidov Yaroslavl State University
%}%

\author{\firstname{T.~A.}~\surname{Puhov}}
\email{alecsandr08062013@gmail.com}
\affiliation{%
P.G. Demidov Yaroslavl State University, 
150003, Russian Federation, Yaroslavl, Sovetskaya Str. 14
}%



%\date{\today}
%\today печатает cегодняшнее число

\selectlanguage{english}

\begin{abstract}
The process of double Compton scattering, $e \gamma \to e \gamma \gamma$, in the presence of
a strongly magnetized, both charge-asymmetric and charge-symmetric electron plasma is considered. 
The amplitude of the process is written out and
the selection rules for photon polarizations are found.
By solving the kinetic equation, an estimate of the efficiency of polarized photon production is obtained.
\end{abstract}


\maketitle


\section{Introduction}

At present, astrophysical objects with scales of magnitudes of 
magnetic field of the order of or greater than its critical value $B_e = m^2/e \simeq 4.41\times10^{13}$ G
(the natural system of units is used, where $c = \hbar = k_{\rm{B}} = 1$, $m$ is the electron mass,
$e>0$ is the elementary charge, $\alpha$ is the fine structure constant) are very interesting.
This class of objects includes radio pulsars and so-called magnetars,
which are neutron stars with magnetic fields ranging from $10^{12}$ G
(radio pulsars) up to $4\times 10^{14}$ G (magnetars)~\cite{Olausen:2013bpa}.
Analysis of emission spectra of these objects
also indicates the presence of electron-positron plasma in the shells of radio pulsars and magnetars with
the minimum density in the magnetospheres is of the order of the Goldreich-Julian density~\cite{GJ:1969}.

One of the fundamental problems in the physics of strongly magnetized neutron stars is the description
features of the observed spectra in the region of X-ray and gamma frequencies, apparently due to
the influence of scattering and absorption of photons in the process of radiation transfer in 
a strongly magnetized plasma.

It is well known (see, for example,~\cite{Suleimanov:2007it}) that Compton scattering, 
$e \gamma \to e \gamma$, is the main process
which is taken into account when solving the problem of radiation transfer. 
However, the total number of photons does not change during scattering.
Previous investigations of radiation transfer problem in strongly
magnetized  plasma~\cite{RChSt:2012,RumChistShlen:2016}
 showed that the reaction photon splitting, $\gamma \to \gamma \gamma$, can play an important 
role as a mechanism of photon production. On the other hand, as was shown in~\cite{RChSt:2012}, 
in the photon energy range $\omega \ll m$, some splitting channels turn out to be 
kinematically closed. Therefore, there is a need
consider alternative mechanisms for changing the number of photons. In a highly magnetized plasma, one
of such mechanisms is the so-called double Compton scattering,
$e \gamma \to e \gamma \gamma$.


Previously, the process $e \gamma \to e \gamma \gamma$ was considered only in plasma without a 
magnetic field~\cite{LightmanApJ1981, Kellner1984} and almost degenerate strongly 
magnetized plasma~\cite{RumChistPuh2023}. In the general case of arbitrarily magnetized plasma, 
the analysis of the photon 
absorption coefficient in the process of double Compton scattering is a rather cumbersome task. However, 
for a nonrelativistic charge-symmetric magnetized plasma with a temperature $T \ll m$ ($\mu =0$ is the 
chemical potential of electrons),
characteristic of the magnetosphere of neutron stars, even at fields $B \sim 10^{12}$ G, electrons 
will occupy
predominantly the ground Landau level, which significantly simplifies the calculations. 
In this paper, we generalize the previously obtained results for the process 
$e \gamma \to e \gamma \gamma$ to the case of a charge-symmetric, strongly magnetized, non-relativistic 
plasma and solve the kinetic equation taking into account the change in the number of photons of 
a certain polarization.


\section{Dispersion properties of a photon in a magnetized medium}

We begin to consider  the process $e \gamma \to e \gamma \gamma$  
with investigation of the photon dispersion properties.  
 The propagation of an electromagnetic radiation in
any active medium is conveniently described in terms
of normal modes (eigenmodes). In turn, the polarization and dispersion properties of the normal modes are
defined by the eigenvectors $r_\alpha^{(\lambda)}(q)$ and eigenvalues $\cal{P}^{(\lambda)}$ of 
the polarization operator ${\cal P}_{\alpha \beta}$, respectively. 
 For further analysis of these properties, it is convenient to expand the tensor ${\cal P}_{\alpha \beta}$ 
in terms of a basis of
4-vectors constructed from the dimensionless tensor of the electromagnetic
field and the 4-vector of the photon momentum $q_\alpha$:
%
\beq
\label{eq:basis}
b_{\mu}^{(1)} = (\varphi q)_\mu, \qquad
 b_{\mu}^{(2)} = (\tilde \varphi q)_\mu, 
\\
\nonumber
b_{\mu}^{(3)} = q^2 \, (\varphi \varphi q)_\mu - q_\mu \, q^2_{\mbox{\tiny $\bot$}}, 
\qquad b_{\mu}^{(4)} = q_\mu, 
\eeq 

\noindent which are eigenvectors of the polarization operator in a constant uniform magnetic field. 
In this case, $(b^{(1)} b^{*(1)}) = -q^2_{\mprp}$, 
$(b^{(2)} b^{*(2)}) = -q^2_{\mprl}$, $(b^{(3)} b^{*(3)}) = -q^2 q^2_{\mprl} 
q^2_{\mprp}$, $(b^{(4)} b^{*(4)}) = q^2$.

We have:
%
\beq
\label{eq:Pab10}
{\cal P}_{\alpha \beta} = \sum_{\lambda = 1}^{3} 
 {\cal P}^{(\lambda)} \, \frac{r_{\alpha}^{(\lambda)} 
(r_{\beta}^{(\lambda)})^{*}}{(r^{(\lambda)})^2} \, , \quad 
r_{\beta}^{(\lambda)} = \sum\limits_{i = 1}^{3} A_i^{(\lambda)} \, b_{\beta}^{(i)} \, , 
\eeq
\noindent where $A_i^{(\lambda)}$ are some complex coefficients.


Here the four-vectors with indices $\bot$ and $\parallel$
belong to the Euclidean \{1,~2\}-subspace and the Minkowski
\{0,~3\}-subspace correspondingly in the frame where the magnetic
field is directed along the third axis; 
$(ab)_{\mprp} = (a \varphi \varphi b) = a_\alpha \varphi_{\alpha}^{\, \rho} \varphi_{\rho \beta} b_\beta$, 
$(ab)_{\mprl} = (a \tilde \varphi \tilde \varphi b) = 
a_\alpha \tilde \varphi_{\alpha}^{\, \rho} 
\tilde \varphi_{\rho \beta} b_\beta$.  
The tensors
$\varphi_{\alpha \beta} =  F_{\alpha
\beta} /B$ and
${\tilde \varphi}_{\alpha \beta} = \frac{1}{2} \varepsilon_{\alpha \beta
\mu \nu} \varphi_{\mu \nu}$ are the dimensionless field tensor and dual
field tensor correspondingly.

The solution of this problem for an arbitrarily magnetized plasma presents significant computational 
difficulties and was previously solved for the case when the leading parameters of the plasma are such 
as temperature, chemical potential, etc. (see, for example~\cite{Potekhin2004}).
 On the other hand, in the case of a strongly magnetized plasma, when the magnetic field is the
largest parameter of the problem,
$eB \equiv \beta \gg m^2,\, \mu^2, \, T^2, \, q^2_{\mprl}$, in the region 
$q^2_{\mprl} \ll (m+\sqrt{m^2+2 \beta})^2$,
using
the results of the works~\cite{Shabad1988, Mikheev2014}
 for ${\cal P}_{\alpha \beta}$, we can obtain the following expansion in
inverse powers of the magnetic field:
%
\beq
\nonumber
&&{\cal P}_{\alpha \beta}  
 \simeq 
 - \frac{2\alpha}{\pi} \; \beta \, {\cal D} \, 
\frac{(\tilde \varphi q)_\alpha (\tilde \varphi q)_\beta}{q^2_{\mprl}} 
+ 
\frac{\alpha}{3\pi}\; (\varphi q)_\alpha (\varphi q)_\beta +
%\\
%\nonumber
%&&+
\frac{\ii \alpha}{\pi} \, \Delta N \, \big [
\varphi_{\alpha \beta} \, (qu) + (q\varphi)_{\alpha} u_{\beta} - 
\\
\label{eq:Pab1}
&&-
(q\varphi)_{\beta} u_{\alpha} \big ]  +
 \frac{\alpha}{3\pi} \, {\cal V} \, \left (q^2 \, g_{\alpha \beta} - 
q_{\alpha} \, q_{\beta} \right )  + 
O \left (\frac{1}{\beta} \right) \, , 
\eeq  
\noindent where 
%
\beq
\label{eq:PabD}
{\cal D} = - {\cal J} (q_{\mprl})  - 
H \left (\frac{q^2_{\mprl}}{4m^2} \right)  \, , 
\eeq
%
\beq
\label{eq:PabJ}
{\cal J} (q_{\mprl}) = 2q^2_{\mprl} \, m^2 \, \int\limits_{-\infty}^{\infty}  \frac{\dd p_z}{E} \, 
\frac{f_{-}(p) + f_{+}(p)}{q_{\mprl}^4 - 4(pq)^2_{\mprl}} \, , 
\eeq
%
\beq
\label{eq:fermidist}
&&f_{\pm}(p) = \frac{1}{1+\exp{[((pu)_{\mprl} \pm \mu)/T]}} \, , 
%\\ [3mm]
%\nonumber
%&&
\quad (pu)_{\mprl} = E u_0 - p_z u_z \, , \quad E=\sqrt{p_z^2+m^2} \, .
\eeq
\noindent Here $u^{\mu}$ is the four vector of plasma velocity and 
the upper sign corresponds to the electron components of the plasma, and the 
lower sign to the positron components, 

%
\beq
\label{eq:H0}
\nonumber
%\nonumber
&&H(z)=\frac{1}{\sqrt{z(1 - z)}} \, \arctg \sqrt{\frac{z}{1 - z}} - 1, \quad 0 \leqslant z \leqslant 1 \, ,
\label{eq:H1}
%H(z) &=& - \frac{1}{2\sqrt{z(z-1)}}
%\ln  \frac{\sqrt{z} + \sqrt{z-1}}{\sqrt{z} - \sqrt{z-1}}  - 1 + 
%\\[3mm]
%\nonumber
%&+& \frac{i\pi}{2\sqrt{z(z-1)}}, \quad z > 1, 
\eeq
%
\beq
\nonumber
\Delta N &=& \int\limits_{-\infty}^{\infty}  \frac{\dd p_z}{E} 
\, (pu)_{\mprl} \, \left [f_{-}(p) - f_{+}(p) \right] = 
\frac{(2 \pi)^2}{\beta} \, (n_{e} - n_{e^+}) \, ,
\label{eq:PabA}  
%\\
%\nonumber 
%&&f((pu),\mu) = \left [\exp{((pu)-\mu)/T} + 1 \right ]^{-1}  \, , \quad   E=\sqrt{p^2+m^2} \, , 
\eeq
\noindent $n_{e}$ $(n_{e^+})$ is the electron (positron) density,
%
\beq
\label{eq:Lambda}
{\cal V} = \ln{(B/B_e)} - 1.792 + 
\frac{3}{2} \, \int\limits_0^1 \dd x \, (1-x^2) \, 
\ln{\left [1- \frac{q^2}{4m^2} \, (1-x^2) \right ]} \, .
\eeq

In the plasma rest frame $(pu)_{\mprl} = E$ and for the expansion of the 
eigenvectors $r^{(\lambda)}_{\alpha}$ in inverse powers of the field we obtain:
%
\beq
\label{eq:r13}
&&r^{(1,3)}_{\alpha} = \left [\mp \sqrt{q^4_{\mprp} + 
(6 \Delta N \, \omega)^2\, \frac{q^2}{q_{\mprl}^2}}  - q^2_{\mprp} \right ]\, 
b^{(1)}_{\alpha} - 
\ii \, \frac{6 \Delta N \, \omega}{q_{\mprl}^2} \,  b^{(3)}_{\alpha} +
\\
\nonumber
&&+
 \ii \,\frac{\Delta N \, k_z \, q^2_{\mprp}}{2\beta \, {\cal D} \, q^2_{\mprl}}\, 
\left [\pm \sqrt{q^4_{\mprp} + 
(6 \Delta N \, \omega)^2\, \frac{q^2}{q_{\mprl}^2}} + q^2_{\mprp} \right ]\; b^{(2)}_{\alpha} + 
O \left (\frac{1}{\beta^2} \right) \, ,
\eeq
%
\beq
\label{eq:r2}
r^{(2)}_{\alpha} =  b^{(2)}_{\alpha} - 
\ii \, \frac{\Delta N  \, k_z }{2\beta \, {\cal D}} \, b^{(1)}_{\alpha} + 
O \left (\frac{1}{\beta^2} \right)  \, .
\eeq

The corresponding eigenvalues of ${\cal P}^{(\lambda)}$ in the approximation
$O(1/\beta)$ are written as follows:
%
\beq
\label{eq:kappa13}
{\cal P}^{(1,3)} = \frac{\alpha}{3\pi} \, q^2 \, {\cal V} + 
\frac{\alpha}{6\pi} \, \left [ 
 \mp \sqrt{q^4_{\mprp} + 
(6 \Delta N \, \omega)^2\, \frac{q^2}{q_{\mprl}^2}}  - q^2_{\mprp} \right ]  
+ O \left (\frac{1}{\beta} \right)  \, ,
\eeq
%
\beq
\label{eq:kappa2}
{\cal P}^{(2)} = \frac{\alpha}{3\pi} \, q^2 \, {\cal V} + \frac{2 \alpha}{\pi} \, \beta \, {\cal D}  + 
 O \left (\frac{1}{\beta} \right) \, .
\eeq

In the case of charge-symmetric plasma $\Delta N = 0$, and in a cold, almost degenerate, 
moderately relativistic plasma, under the condition
$\Delta N /(2m) \simeq v_F \ll 1$, where  $v_F$ is the Fermi velocity,
eigenvectors~(\ref{eq:r13}) -- (\ref{eq:r2}) and
eigenvalues~(\ref{eq:kappa13}) -- (\ref{eq:kappa2}) will take the form:

%
\beq
\label{eq:r132}
r^{(1)}_{\alpha} =  -2 q^2_{\mprp} \, b^{(1)}_{\alpha}  + O \left (\frac{1}{\beta^2} \right) \, ,
 \quad 
%\eeq
%
%\beq
%\label{eq:r20}
r^{(2)}_{\alpha} =  b^{(2)}_{\alpha}  + O \left (\frac{1}{\beta^2} \right)  \, .
\eeq


%
\beq
\label{eq:kappa10}
&&{\cal P}^{(1)} = \frac{\alpha}{3\pi} \, q^2 \, {\cal V} - \frac{\alpha}{3\pi} \, 
q^2_{\mprp} + O \left (\frac{1}{\beta} \right)  \, ,
\\[3mm]
\label{eq:kappa30}
&&{\cal P}^{(3)} = \frac{\alpha}{3\pi} \, q^2 \, {\cal V} + O \left (\frac{1}{\beta} \right)  \, ,
\eeq
%
\noindent and the eigenvalue ${\cal P}^{(2)}$ is determined by the formula~(\ref{eq:kappa2}). 
Vector $r^{(3)}_{\alpha} =  O \left (1/\beta^2\right)$, and, therefore, under the 
conditions under consideration, cannot participate in the description of the polarization 
properties of the photon.

It should be noted that the obtained eigenvectors~(\ref{eq:r132}) % and~(\ref{eq:r20}) 
of the polarization 
operator in a charge-symmetric plasma are not normalized. Therefore, to describe the polarization states of 
photons in such a plasma, it is convenient to introduce normalized vectors

%
\beq
\label{eq:epsilon}
%&&
\varepsilon_\alpha^{(1)}(q) = \frac{r^{(1)}_{\alpha}}{\sqrt{|(r^{(1)} r^{*(1)})|}} = 
\frac{(q \varphi)_\alpha}{\sqrt{q_{\mprp}^2}} \, , \quad
%\\
%\nonumber
%&&
\varepsilon_\alpha^{(2)}(q) = \frac{r^{(2)}_{\alpha}}{\sqrt{|(r^{(2)} r^{*(2)})|}} = 
\frac{(q \tilde \varphi)_\alpha}{\sqrt{q_{\mprl}^2}}.
\eeq
\noindent Here symbols 1 and 2 correspond to the
$X$ - and $O$ - modes commonly used in the literature~(see, for example,~\cite{Mushtukov2016}).

It is easy to see that the polarization states introduced in this way in a magnetized 
plasma, with an accuracy of $O(1/\beta^2)$ and $O(\alpha^2)$, have the same form as in a magnetized
vacuum~\footnote{The term <<magnetized vacuum>>
means a magnetic field without plasma.}. This conclusion agrees with the results of the 
work~\cite{Shabad1988}, and in the limit of $\omega \ll m$, after appropriate transformations 
(selection of the spatial
part of the vector $\varepsilon^{(2)}_\alpha$ and transition to a coordinate system where 
the photon momentum vector
is directed along the $z$ axis), with the results of the work~\cite{Potekhin2004}.


Moreover, in the $O(1/\beta)$ approximation, the dispersion law of the photon of mode 1 is 
practically no different from the vacuum law, $q^2 \simeq 0$. Indeed, from the dispersion equation 
for this mode
%
\beq
q^2 - {\cal P}^{(1)} = 0 \, 
\label{disper1}
\eeq
\noindent and from the formula~(\ref{eq:kappa10}) it follows that
%
\beq
q_{\mprl}^2 = \left (1- \frac{\alpha}{3\pi} \frac{1}{1-\frac{\alpha}{3\pi} \cal V} \right) \, q_{\mprp}^2  
\simeq q_{\mprp}^2 \left (1- \frac{\alpha}{3\pi} \right)\, , 
\label{disper12}
\eeq
\noindent so that $q^2 \simeq 0$,
while remaining negative. In addition, it follows from formula~(\ref{eq:kappa10}) that in
the region $q^2_{\mprl} \ll (m+\sqrt{m^2+2 \beta})^2$ the eigenvalue of the polarization
operator of the photon corresponding to mode 1 has no imaginary part~\footnote{Strictly speaking, 
the imaginary part
turns out to be strongly suppressed, compared to the real part, so that such a photon will be quasi-stable
(see, for example,~\cite{Yarkov2022}).}, which leads to the impossibility
of the production of an $e^+e^-$ pair by such a photon in this region.

On the other hand, the dispersion properties of the mode 2 photon undergo significant changes even
in comparison with a magnetized vacuum and, therefore, will have an additional effect
on the kinematics of processes involving photons of this mode. Let us consider this statement in more detail.
In Fig.~\ref{fig:disT}, as an illustration, the dispersion laws of the mode 2 photon are presented 
for the case when the photon propagates across the magnetic field,
as solutions of the equation
%
\beq
q^2 - {\cal P}^{(2)} = 0 \, 
\label{disper}
\eeq
\noindent in charge-symmetric ($\mu=0$) magnetized plasma for different temperature values.
As is easy to see, in contrast to a pure magnetic field, in plasma there is a region with $q^2 > 0$ 
below the kinematic threshold of $e^+e^-$ pair production, determined by the condition $q^2_{\mprl} = 4 m^2$.

This fact is associated with the appearance of a plasma frequency in the presence of real electrons and 
positrons of the medium, which can be determined from the equation
%
\begin{equation}
\omega_{pl}^2 - {\cal P}^{(2)} (\omega_{pl}, {\mathbf k} \to 0 ) = 0.
\label{eq:omegapl}
\end{equation}

In the case of a strongly magnetized charge-symmetric, non-relativistic plasma, equation~(\ref{eq:omegapl}) 
can be solved approximately. As a result, we obtain the classical result 
$\omega_{pl}^2 \simeq 2(4\pi \alpha n_{e})/m$ (the factor 2 is due to the equality of 
the number of electrons and positrons), where
%
\begin{eqnarray}
n_{e} \simeq \beta \sqrt{\frac{mT}{(2\pi)^3}}\,e^{-m/T}.
\label{eq:ne}
\end{eqnarray}
\noindent Thus, the $\omega_{pl}$ for charge-symmetric plasma is exponentially suppressed.
 On the other hand, in charge-asymmetric plasma such suppression is absent and
$\omega_{pl}^2 \simeq (2\alpha \beta/\pi)v_F$.

However, even in a charge-symmetric plasma for a temperature of $T=50$ keV and a magnetic 
field of $B=200 B_e$ we obtain the following estimate: $\omega_{pl} \simeq 3$ keV, which can affect 
the kinematics of various photon processes.
For example, for one of the main reactions for the production of polarized photons, which is the process 
of splitting a photon into two photons, $\gamma \to \gamma \gamma$, the presence of a plasma frequency 
leads to new selection rules for polarizations: in the region below the threshold for the production of
$e^+e^-$ pairs, $q^2_{\mprl} = 4 m^2$ and in the region $q^2 > 0$, the splitting channels responsible 
for the production of mode 2 photons,
$\gamma_2 \to \gamma_2 \gamma_2$,
$\gamma_1 \to \gamma_2 \gamma_2$ and $\gamma_1 \to \gamma_1 \gamma_2$ are kinematically closed. 
In this region, only the channel $\gamma_2 \to \gamma_1 \gamma_1$
responsible for the production of mode 1 photons is allowed. Therefore, in this case, the leading 
channel for the production of mode 2 photons will be the double Compton scattering process, 
$e \gamma_2 \to e \gamma_2 \gamma_2$.



\section{Amplitude of double Compton process}

The amplitude of the process $e \gamma \to e \gamma \gamma$ in the tree approximation in 
the case when all electrons 
occupy the ground Landau level can be obtained from the results of the work~\cite{RumChistPuh2023} 
and presented in the following form:
%
\beq
\label{eq:ampl1}
{\cal M}_{\lambda \to \lambda' \lambda''} =&& (4\pi \alpha)^{3/2}
\exp \left [-\frac{i Q_x}{2 \beta}\Big( p_y+p'_y\Big) \right ]
\exp \left [-\frac{q_{\mprp}^2+\prp{q}'^2 + q_{\mprp}^{\prime \prime \, 2}}{4 \beta} \right ]
\times
\\
\nonumber
\times &&
\exp \left [\frac{i}{2 \beta}\Big( (q \varphi q')+(q \varphi q'')+(q'' \varphi q')\Big) \right ]
\frac{\varepsilon^{(\lambda)}_{\alpha}(q)\varepsilon^{*(\lambda')}_{\beta}(q')
\varepsilon^{*(\lambda'')}_{\gamma}(q'')\,T_{\alpha \beta \gamma}}
{\left (q_{\mprl}^2 + 2(pq)_{\mprl} \right )
\left (q_{\mprl}^{\prime \prime \, 2} + 2(p'q'')_{\mprl}\right )}  +
\\
\nonumber
+&&(\mbox{photons momenta permutations}),
\eeq

\noindent where
\beq
\nonumber
T_{\alpha \beta \gamma} = &&\frac{2m}{\sqrt{-Q_{\mprl}^2}}\bigg \{ \varkappa^2 
(Q \tilde \varphi)_\alpha (Q \tilde \varphi)_\beta (Q \tilde \varphi)_\gamma - 
 \varkappa \Big [(Q \tilde \varphi)_\gamma \big ((Q \tilde \Lambda)_\alpha (q \tilde \varphi)_\beta + 
(Q \tilde \varphi)_\alpha (q \tilde \Lambda)_\beta + 
\\
\nonumber
+&&
(Q \tilde \varphi)_\alpha (q'' \tilde \Lambda)_\beta \big) + 
(Q \tilde \varphi)_\alpha (q'' \tilde \varphi)_\beta (Q \tilde \Lambda)_\gamma
- Q^2_{\mprl} \tilde \varphi_{\alpha \gamma} (Q \tilde \varphi)_\beta \Big]  + 
\\
\nonumber
+&&(Q \tilde \varphi)_\gamma 
\Big [(q''\tilde \varphi)_\alpha (q \tilde \varphi)_\beta + 
(q''\tilde \Lambda)_\alpha (q \tilde \Lambda)_\beta - (Q \tilde \Lambda)_\alpha 
(q'' \tilde \Lambda)_\beta \Big ]+
\\
\nonumber
+&& (Q \tilde \Lambda)_\gamma 
\Big [(q''\tilde \Lambda)_\alpha (q \tilde \varphi)_\beta + (q''\tilde \varphi)_\alpha 
(q \tilde \Lambda)_\beta + 
(Q\tilde \Lambda)_\alpha (q' \tilde \varphi)_\beta  
+ (Q \tilde \varphi)_\alpha (q \tilde \Lambda)_\beta\Big ] 
\bigg \} \, ,
\eeq
\noindent $Q_{\alpha} = (q - q' - q'')_{\alpha}$, $\varkappa = \sqrt{1 - 4m^2/Q^2_{\mprl}}$ and
$Q^2_{\mprl} = (q - q' - q'')^2_{\mprl} <0$,
$q_{\alpha} = (\omega,{\bf k})$, $q'_{\alpha} = (\omega',{\bf k'})$  and  $q''_{\alpha} = (\omega'',{\bf k''})$
are the four-momenta of the initial and final photons correspondingly, $p_y$ and $p'_y$ are the 
momenta components of the initial and final electrons correspondingly.

Substituting into (\ref{eq:ampl1}) the polarization vectors from (\ref{eq:epsilon}), 
in the limit $\omega, \, \omega', \, \omega'' \ll m \ll \sqrt{\beta}$ we obtain the following 
expression for the amplitude of the leading scattering channel in the <<symmetric>> form
%
%
\beq
\label{eq:amp222}
{\cal M}_{2 \to 22} \simeq  - \frac{2 (4 \pi \alpha)^{3/2}}{m} \; 
\frac{\sqrt{q^2_{\mprl} q^{\prime \, 2}_{\mprl} 
q^{\prime \prime \, 2}_{\mprl}}}{\omega \omega^{\prime} 
\omega^{\prime \prime}} \;  \, \left \{\frac{q_z}{\omega} + 
\frac{q_z^{\prime}}{\omega^{\prime}} + \frac{q_z^{\prime \prime}}{\omega^{\prime \prime}}\right \} \, .
\eeq
\noindent where $\theta$, $\theta'$ and $\theta''$ 
are the angles between photon momenta, ${\bf k}$, ${\bf k}'$ and ${\bf k}''$ and
the direction of the magnetic field, respectively.


\section{Efficiency of photon production in a strongly magnetized medium}


 To analyze the efficiency of mode 2 photon production, we write the kinetic equation taking into 
account the processes of Compton and double Compton scattering in the following form: 
\footnote{Without a magnetic field a similar 
equation was obtained in~\cite{Kellner1984}.}
%
\begin{eqnarray}
\label{eq:kinet1}
\frac{\partial f_{\omega, \theta}}{\partial t} = St_{scat}f_{\omega, \theta} + St_{cr}f_{\omega, \theta} \, .
\end{eqnarray}
\noindent Here
%
\begin{eqnarray}
\label{eq:kinet2}
&&St_{scat}f_{\omega,\theta} = \int dW_{2\to 2} \big \{[1-f_{-}(p)]f_{-}(p')
(1+f_{\omega, \theta})f_{\omega', \theta'} - 
\\
\nonumber
&& -
 [1-f_{-}(p')]f_{-}(p)(1+f_{\omega', \theta'})f_{\omega, \theta}\big \}
\end{eqnarray}
\noindent is the collision integral due to Compton scattering in a strongly magnetized plasma,
%
\begin{eqnarray}
\label{eq:kinet3}
dW_{2\to2} \simeq \frac{\beta}{16 (2\pi)^4
\omega}
\mid {\cal M}_{2\to 2}\mid^2 \,
\delta (\omega + E - \omega' - E') 
\frac{dp_z\,d^3 k^{'} d^3 k^{''}}{ E E' \omega'} \, ,
\end{eqnarray} 
\noindent where $E = \sqrt{p_z^2 + m^2}$ and $E' = \sqrt{(p_z + q_z - q'_z)^2
+ m^2}$ are the energies of the initial and final electrons correspondingly and $f_{-}(p)$  $(f_{-}(p'))$
are determined by the formula~(\ref{eq:fermidist}), $f_{\omega, \theta}$ is a nonequilibrium 
photon distribution function,
%
\begin{eqnarray}
\label{eq:kinet4}
{\cal M}_{2\to 2} \simeq -8 \pi \alpha \frac{\sqrt{q^2_{\mprl} q^{\prime \, 2}_{\mprl}}}
{\omega \omega^{\prime}}
\end{eqnarray} 
\noindent is the scattering amplitude of mode 2 photons in the limit 
$\omega, \, \omega' \ll m \ll \sqrt{\beta}$~\cite{RCh09},
%
\begin{eqnarray}
\label{eq:kinet5}
&&St_{cr}f_{\omega, \theta} = \int dW_{2\to 22} \big \{[1-f_{-}(p)]f_{-}(p')
(1+f_{\omega, \theta})f_{\omega' ,\theta'}f_{\omega'', \theta''} - 
\\
\nonumber
&&-[1-f_{-}(p')]f_{-}(p)(1+f_{\omega',\theta'})(1+f_{\omega'', \theta''})f_{\omega, \theta}\big \}
\end{eqnarray}
\noindent is the collision integral due to double Compton scattering,
%
\begin{eqnarray}
\label{eq:abs1}
dW_{2\to 22} \simeq \frac{\beta}{64 (2\pi)^7
\omega}
 \mid {\cal M}_{2\to 22}\mid^2  \,
\delta (\omega + E - \omega' -\omega''- E') 
\frac{dp_z\,d^3 k^{'} d^3 k^{''}}{ E E' \omega' \omega''} \, ,
\end{eqnarray}
\noindent where $E = \sqrt{p_z^2 + m^2}$ and
$E' = \sqrt{(p_z - k_z - k'_z - k''_z)^2 + m^2}$ -- energies of initial and final electrons
correspondingly and the amplitude ${\cal M}_{2\to 22}$ is determined by the formula~(\ref{eq:amp222}).


Before looking for a solution to the equation~(\ref{eq:kinet1}), let's make the following assumptions.
Since the scattering cross section of mode 1 photons in a strongly magnetized plasma is suppressed by 
the  inverse value of the magnetic field, and hence their mean free path significantly exceeds the mean 
free path of mode 2 photons (see, for example~\cite{RChSt:2012,RCh09}), then mode 1 photons can 
freely escape from the region 
(scale of the mean free path of mode 1) occupied by the plasma. Therefore, we can ignore the processes in which mode 1 
is involved, even if $\omega_{pl} \to 0$.

In addition, the second term on the right-hand side of the 
equation~(\ref{eq:kinet1}) is obviously much smaller than the first, so we 
can search for a solution to the equation~(\ref{eq:kinet1}) in the form of successive approximations:
%
\beq
\label{eq:solv}
f_{\omega, \theta} = f_{\omega, \theta}^{(0)} + f_{\omega, \theta}^{(1)} + \ldots \, .
\eeq

Substituting~(\ref{eq:solv}) in~(\ref{eq:kinet1})  and integrating over $d^3k/(2\pi)^3$, 
taking into account the conservation of the number of photons in 
the Compton process, we obtain the number of photons of mode 2 produced per unit volume, per unit time 
in the form
%
\beq
\label{eq:photnumb}
\frac{d N}{d V d t} = \int \frac{d^3 k}{(2\pi)^3} St_{cr}f_{\omega, \theta}^{(0)} \, , 
\eeq
\noindent where $f_{\omega, \theta}^{(0)}$ is the solution to the equation
%
\beq
\label{eq:solv0}
\frac{\partial f_{\omega, \theta}^{(0)}}{\partial t} = St_{scat}f_{\omega, \theta}^{(0)} \, . 
\eeq



It should also be noted that mode 2 photons will fairly quickly come into equilibrium with the electron plasma, 
so that the distribution function can be considered almost isotropic, 
$f_{\omega, \theta}^{(0)} \simeq f_{\omega}^{(0)}$ and does not depend on time. In this case, in the limit 
of non-relativistic plasma the equation~(\ref{eq:solv0}) becomes a stationary 
Kompaneets-type equation~\cite{Kompaneets1957}:
%
\beq
\label{eq:komp}
 \frac{\partial f_{\omega}^{(0)}}{\partial x} + f_{\omega}^{(0)} (1 + f_{\omega}^{(0)}) = \frac{Q}{x^4} \, ,
\eeq
\noindent where $x=\omega/T$, $Q$ is a flow of photons in momentum space.

The equation~(\ref{eq:komp}) has a particular solution~\cite{Dub2009}:
%
\beq
\label{eq:statsolv}
f_{\omega, Q}^{(0)} = \frac{1-x}{2 x} - \frac{4x}{(x^2-1)^2} 
\frac{\mathrm{HeunD}^{\prime}\left(0,\, -Q-1/2, \, 2Q -1/2,\,-Q,\,y \right)}
{\mathrm{HeunD}\left(0,\, -Q-1/2, \, 2Q -1/2,\,-Q,\,y \right)} \, ,
\eeq
\noindent where $\mathrm{HeunD}\left(a,\, b, \, c,\,d,\,y \right)$ is the double confluent Heun 
function of the argument $y$, satisfying the equation
\beq
\label{eq:heuneq}
\mathrm{HeunD}^{\prime \prime} - \frac{a (y^2+1)-2y(y^2-1)}{(y^2-1)^2} \, \mathrm{HeunD}^{\prime} + 
\frac{b y^2 + (c+2a)y+d}{(y^2-1)^3} \, \mathrm{HeunD}   = 0 \, ,
\eeq
\noindent $\mathrm{HeunD}^{\prime} \equiv \frac{d\mathrm{HeunD}\left(a,\, b, \, c,\,d,\,y \right)}{dy}$, 
$y=(x^2+1) (x^2-1)^{-1}$.

Substituting~(\ref{eq:kinet5}) into~(\ref{eq:photnumb}), taking into account~(\ref{eq:amp222}), 
after simple integration, in the leading approximation over $\omega_{pl}$, we obtain
%
\beq
\label{eq:photnumb1}
\frac{d N}{d V d t} =  \frac{8\alpha^3 n_e T^5}{45 \pi^2 m^4} J(Q) \, ,
%\simeq 1.5 \cdot 10^{16} \, \left(\frac{1}{\mbox{cm}^3\, \mbox{c}} \right) \, 
%\left(\frac{n_e}{3 \cdot 10^{13}\,\mbox{cm}^{-3}} \right) \, \left(\frac{T}{5\, \mbox{keV}} \right)^5 J(Q)\, ,
\eeq
where 
%
\beq
\label{eq:J1}
J(Q) = \int\limits_{0}^{\infty} dx\, x 
\int\limits_{0}^{x} dx' \, x' (x-x') \big \{f_{\omega',Q}^{(0)}f_{\omega-\omega',Q}^{(0)} -f_{\omega,Q}^{(0)} -2
f_{\omega,Q}^{(0)} f_{\omega',Q}^{(0)} \big \} \, , \quad  x'=\omega'/T \, . 
\eeq
Taking into account the solution~(\ref{eq:statsolv}), the integral~(\ref{eq:J1}) for different values 
of the parameter $Q$ can be estimated numerically. 
 We obtained the following approximation $J(Q)\simeq 17.64 Q$. As a result we get
%
\beq
\label{eq:photnumb2}
\frac{d N}{d V d t} 
\simeq 2.6 \cdot 10^{17} Q \, \left(\frac{1}{\mbox{cm}^3\, \mbox{c}} \right) \, 
\left(\frac{n_e}{3 \cdot 10^{13}\,\mbox{cm}^{-3}} \right) \, \left(\frac{T}{5\, \mbox{keV}} \right)^5 \, .
\eeq

Using the results of the work~\cite{LightmanApJ1981}, we present for comparison an estimate for
the number of photons produced in the process
$e \gamma \to e \gamma \gamma$ in an isotropic plasma without a magnetic field for $T=5$ keV and
$n_e=3 \cdot 10^{13}\,\mbox{cm}^{-3}$:
%
\begin{eqnarray}
\label{eq:numbervac}
\frac{d N^{vac}}{d V d t} \simeq  2 \, \frac{16 \alpha^3 n_e T^5}{3 \pi^2 m^4} \,
\int\limits_{2 \omega_{pl}/T}^{\infty} \frac{d x \, x^4}{e^x-1} 
 \left [\frac{2}{3} \ln{\left(\frac{2 T x}{\omega_{pl}}\right )} - 1 \right] \simeq
 2.8 \cdot 10^{20} \, \left(\frac{1}{\mbox{cm}^3\, \mbox{c}} \right)  \, .
\end{eqnarray}

As can be seen from the obtained results, the efficiency of photon production in plasma without a magnetic field
will be three orders of magnitude higher than in magnetized plasma. This may be due to the fact that
the magnetic field, affecting the amplitude of the double Compton
process~(\ref{eq:amp222}), eliminates the infrared divergence that occurs in the amplitude calculated
without a magnetic field.
Nevertheless, the reaction under consideration can serve as a fairly efficient mechanism
for producing polarized photons in the presence of a strongly magnetized plasma.

\section{Conclusion}

The double Compton process $e\gamma\to e\gamma\gamma$ is considered,
in the presence of a strongly magnetized nonrelativistic plasma.
Under these conditions, changes in the dispersion and polarization properties of photons are investigated.
It is shown that under such conditions,
the photon polarization vectors will remain the same as in a pure magnetic field.
Analysis of the dispersion law showed that in a cold, strongly magnetized plasma,
the process of double Compton scattering can become an effective mechanism for obtaining polarized photons.
As a consequence, this fact can affect the formation of the emission spectra of
strongly magnetized neutron stars.
An estimate is obtained for the number of mode 2 photons produced in the process
$e \gamma \to e \gamma \gamma$ in the magnetosphere of a strongly magnetized neutron star. It is shown that 
the double Compton scattering process can be an efficient mechanism for producing polarized photons in the 
presence of a strongly magnetized plasma.



%
% Список литературы
%
\begin{thebibliography}{99}
\bibitem{Olausen:2013bpa}
%\refitem{article}
S.~A.~Olausen and  V.~M.~Kaspi, Astrophys. J. Suppl. \textbf{212}, 6 (2014).
%
\bibitem{GJ:1969}
%\refitem{article}
P.~Goldreich and W.~H.~Julian, Astrophys. J. \textbf{157}, 869 (1969). 
%
\bibitem{Suleimanov:2007it}
%\refitem{article}
V.~Suleimanov and K.~Werner, Astron. Astrophys. \textbf{466}, 661 (2007).
%
\bibitem{RChSt:2012}
%\refitem{article}
M.~V.~Chistyakov, D.~A.~Rumyantsev, and N.~S.~Stus', Phys. Rev. D  \textbf{86}, 043007 (2012).
%
\bibitem{RumChistShlen:2016}
%\refitem{article}
M.~V.~Chistyakov, D.~A.~Rumyantsev, and D.~M.~Shlenev, EPJ Web Conf. \textbf{125}, 04017 (2016).
%\refitem{article}
%
\bibitem{LightmanApJ1981}
A.~P.~Lightman, Astrophys. J. \textbf{244}, 392 (1981).
%\refitem{article}
\bibitem{Kellner1984}
S.~R.~Kel'ner  and E.~S.~Shikhovtseva, Pis'ma Astron. Zh. \textbf{10}, 76 (1984). 
%
\bibitem{RumChistPuh2023}
D.~A.~Rumyantsev, A.~A.~Yarkov, M.~V.~Chistyakov, and T.~A.~Pukhov, 
 Physics of Atomic Nuclei  \textbf{86}, 589 (2023).
%
\bibitem{Potekhin2004}
A.~Y.~Potekhin, D.~Lai, G.~Chabrier, and Wynn~C.~G.~Ho, Astrophys. J. \textbf{612}, 1034 (2004). 
%
\bibitem{Shabad1988}
A.~E.~Shabad, Tr. Fiz. Inst. Akad. Nauk SSSR \textbf{192}, 5 (1988).
%
\bibitem{Mikheev2014}
   N.~V.~Mikheev, D.~A.~Rumyantsev, M.~V.~Chistyakov, 
   Zh.~Eksp.~Teor.~Fiz. \textbf{146}, 289 (2014) [J.~Exp.~Theor.~Phys. \textbf{119}, 251 (2014)].
%
\bibitem{Mushtukov2016}
A.~A.~Mushtukov, D.~I.~Nagirner, and J.~Poutanen, Phys. Rev. D  \textbf{93}, 105003 (2016).
%
\bibitem{Yarkov2022}
A.~A.~Yarkov, D.~A.~Rumyantsev, and  M.~V.~Chistyakov, Physics of Atomic Nuclei \textbf{85}, 1566 (2022).
%
\bibitem{RCh09}
M.~V.~Chistyakov and D.~A.~Rumyantsev, Int. J. Mod. Phys. A  \textbf{24}, 3995 (2009).
%
%\bibitem{KuzRumShlen:2015}
%A.~V.~Kuznetsov, D.~A.~Rumyantsev, and D.~M~Shlenev, Int. J. Mod. Phys. A  \textbf{30}, 1550049 (2015).
%
\bibitem{Kompaneets1957}
A.~S.~Kompaneets, Zh.~Eksp.~Teor.~Fiz. \textbf{4}, 876 (1957).
%
\bibitem{Dub2009}
A.~E.~Dubinov, Pis'ma v Zh. Tekh. Fiz.  \textbf{35}, 25 (2009). 
%[Technical Physics Letters \textbf{35}, 25 (2009)].
\end{thebibliography}
%
%
\newpage
%Рисунок в статью можно включить при помощи окружения figure:
\begin{figure}
\setcaptionmargin{5mm}
\onelinecaptionstrue
\includegraphics{fig_1.eps} % Так вставляется рисунок
\captionstyle{normal}
\caption{Dispersion laws of the mode 2 photon in a strong magnetic field $B/B_e = 200$
and neutral plasma for different temperatures: $T = 1$ MeV
(upper curve), $T = 0.5$ MeV (middle curve), $T = 0.25$ MeV (lower curve).
The dispersion of the photon without plasma is indicated by the dashed line. The diagonal dashed
line corresponds to the vacuum dispersion law, $q^2 = 0$.
The angle between the photon momentum and the direction of the magnetic field is $\pi/2$.}
\label{fig:disT}
\end{figure}



\end{document}



%%%%%%%%%%%%%%%%%%%%%%%%%%%% Character code reference %%%%%%%%%%%%%%%%%%%%%%%%%%
%                                                                             %
%     Upper case russian letters (CP866): АБВГДЕЁЖЗИЙКЛМНОПРСТУФХЦЧШЩЪЫЬЭЮЯ   %                                                                       %
%     Lower case russian letters (CP866): абвгдеёжзийклмнопрстуфхцчшщъыьэюя   %
%                     Upper case letters: ABCDEFGHIJKLMNOPQRSTUVWXYZ          %
%                     Lower case letters: abcdefghijklmnopqrstuvwxyz          %
%                                   Digits: 0123456789                        %
% Square, curly, angle braces, parentheses: [] {} <> ()                       %
%                Backslash, slash, solidus: \ / |                             %
%       Period, interrogative, exclamation: . ? !                             %
%                 Comma, colon, semi-colon: , : ;                             %
%          Underscore, hyphen, equals sign: _ - =                             %
%             Quotes (left, right, double): ` ' "                             %
%     Commercial-at, hash, dollar, percent: @ # $ %                           %
%  Ampersand, asterisk, plus, caret, tilde: & * +   ^                         %
%                                                                             %
%%%%%%%%%%%%%%%%%%%%%%%%%%%%%%%%%%%%%%%%%%%%%%%%%%%%%%%%%%%%%%%%%%%%%%%%%%%%%%%
%
% ****** End of file apssampr.tex ******
