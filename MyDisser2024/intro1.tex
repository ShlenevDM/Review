%\section{Введение}
\intro

В настоящее время является установленным фактом, что наличие магнитного поля в широком классе астрофизических 
объектов представляет типичную ситуацию для наблюдаемой Вселенной. При этом масштаб индукции 
магнитного поля может варьироваться в очень широких пределах: от крупномасштабных ($\sim 100$ килопарсек) межгалактических магнитных полей  
$\sim 10^{-21}$ Гс~\cite{Ryu:2012}, 
до полей, реализующихся в сценарии ротационного взрыва 
сверхновой $\sim 10^{17}$ Гс~\cite{Bisnovatyi-Kogan:1970, Bisnovatyi-Kogan:1989}. 
При этом, особый интерес представляют объекты с 
полями масштаба так называемого критического значения  $B_e = m_e^2/e \simeq 4.41 \times 10^{13}$ Гс~\footnote{В работе 
используется естественная система единиц, где
$c = \hbar = k_{\rm{B}} = 1$, $m_e$ -- масса электрона, $m_f$ -- масса фермиона, 
$e_f$ -- заряд фермиона, $e > 0$ -- элементарный заряд.}. К ним, 
в частности, относятся изолированные нейтронные звезды, включающие в себя радиопульсары и так называемые магнитары, 
обладающими магнитными полями с индукцией от $10^{12}$~Гс (радиопульсары) до $4\times 10^{14}$~Гс
(магнитары). Недавние наблюдения 
позволяют, в частности, отождествить  некоторые  астрофизические объекты, такие как источники 
мягких повторяющихся гамма-всплесков (SGR) и  аномальные рентгеновские пульсары (AXP),   
с магнитарами (для обзора см., например,~\cite{Olausen:2014}).
 Согласно наиболее известной в настоящее время модели~\cite{Thompson:1995,Duncan:1996,Lyutikov:2002}   
в окрестности таких объектов возможно существование  
сильного  магнитного поля, достигающего величины  $10^{15} - 10^{16}$~Гс.

%%%%%%%%%%%%%%%%%%%%%%%%%%%%%%%%%%%%%%%%%%%%%%%%%%%%%%%%%%%%%%%%%%%%%%%%%%%%%%%%%%%%%%%%%%%%%%%%%%%%%%%%%%%%%%%%%%%%%%%%% 

Анализ спектров излучения как радиопульсаров, так и магнитаров свидетельствует также о наличии 
электрон-позитронной плазмы в их магнитосферах с  концентрацией порядка значения концентрации  
Голдрайха-Джулиана~\cite{GJ:1969}: 
%
\begin{eqnarray}
\label{eq:ngj}
n_{GJ} \simeq 
 3\times 10^{13}\, \mbox{см}^{-3} 
\left (\frac{B}{100B_e} \right )\left (\frac{10\,\mbox{сек}}{P} \right ) \, .
\end{eqnarray}                                         

Естественно ожидать, что такие экстремальные условия будут оказывать существенное 
влияние на квантовые процессы, где в конечном или начальном состоянии могут присутствовать 
как электрически заряженные, так и 
электрически нейтральные частицы, например, электроны и фотоны. Кроме того, внешняя 
активная среда может катализировать реакции с участием таких экзотических частиц, как 
аксионы, фамилоны и т.п., что представляет интерес для поиска новой физики за пределами 
стандартной модели. 

% Больше про магнитные поля

Среди квантовых процессов, свойства которых существенно, а иногда
принципиально меняются под воздействием  замагниченной среды, особый интерес 
для астрофизики представляют одно- и двухвершинные процессы. Это обусловлено тем, 
что с точки зрения влияния микрофизических процессов на макроскопические характеристики астрофизических 
объектов (например, скорость потери энергии, число рождаемых частиц, коэффициент диффузии и т.п.)
существенными будут лишь те реакции, которые дают лидирующие по константам связи вклады.

%%%%%%%%%%%%%%%%%%%%%%%%%%%%%%%%%%%%%%%%%%%%%%%%%%%%%%%%%%%%%%%%%%%%%%%%%%%%%%%%%%%%%%%%%%%%%%%%%%%%%%%%%%%%%%%%%%%%%%%%%%%%%%%

% обобщённые амплитуды, двухвершинные процессы

%%%%%%%%%%%%%%%%%%%%%%%%%%%%%%%%%%%%%%%%%%%%%%%%%%%%%%%%%%%%%%%%%%%%%%%%%%%%%%%%%%%%%%%%%%%%%%%%%%%%%%%%%%%%%%%%%%%%%%%%%%%%%%%

Нейтринные процессы в коре нейтронной звезды играют определяющую роль в её остывании на начальной стадии эволюции~\cite{Yakovlev2000}. 
Процесс рассеяния фотона на электроне плазмы, сопровождающийся рождением пары нейтрино-антинейтрино (так называемый 
фотонейтринный процесс), $e\gamma\to e\nu\bar\nu$, был рассмотрен в работах~\cite{RumCh,Skobelev:2000,BorKer2,RumChMik}. 
Выражения для нейтринной светимости (энергии, уносимой нейтринной парой за единицу времени из единицы объёма) были получены 
в~\cite{RumCh,BorKer2} для случаев нерелятивистской и релятивистской плазмы. В них не была учтена анизотропия дисперсии фотона, 
которая оказывает значительное влияние на результат. В работе~\cite{RumChMik} нейтринная светимость была вычислена с учётом дисперсионных 
свойств и перенормировки волновой функции. В ней были проведены расчёты для характеристик плазмы внешней коры магнитара 
($B\sim 10^{14}-10^{16}$ Гс, $T\sim 10^8 - 10^9$ К, $\rho\sim 10^6-10^9$ г/см$^3$). При таких условиях величина $\sqrt{eB}$ 
является максимальным параметром среды и намного превосходит температуру, химический потенциал $\mu$ и энергии фотонов и 
электронов. Полученные в перечисленных работах формулы становятся неприменимыми, когда плотность среды достигает значения 
$\rho = 10^9$, что реализуется на границе внешней и внутренней коры магнитара. При этом начинают возбуждаться высшие уровни 
Ландау виртуального электрона, что приводит к возникновению возможности для резонанса.

%%%%%%%%%%%%%%%%%%%%%%%%%%%%%%%%%%%%%%%%%%%%%%%%%%%%%%%%%%%%%%%%%%%%%%%%%%%%%%%%%%%%%%%%%%%%%%%%%%%%%%%%%%%%%%%%%%%%%%%%%%%%%%%

Другой важнейший процесс во взаимодействии радиации с веществом для множества астрофизических объектов 
- это комптоновское рассеяние, $e\gamma\to e\gamma$. Он был рассмотрен многочисленными 
авторами~\cite{Canuto1971,Gnedin1974,Blandford1976,Borner1979,DeRaad1974,Herold:1979,Harding:1986,Bussard1986,
Wasserman1980,Harding1991,Graziani1993,Harding1995,Gonthier:2014wja,Mushtukov:2015qul}.
Основное влияние магнитного поля заключается в 
появлении резонансов, соответствующих переходам виртуального электрона на возбуждённые уровни Ландау. 
При этом эффективное сечение рассеяния может превышать томпсоновское значение $\sigma_T = 8\pi\alpha^2 / (3m_e^2)$ 
до $10^6$ раз~\cite{Mushtukov:2015qul}.

%%%%%%%%%%%%%%%%%%%%%%%%%%%%%%%%%%%%%%%%%%%%%%%%%%%%%%%%%%%%%%%%%%%%%%%%%%%%%%%%%%%%%%%%%%%%%%%%%%%%%%%%%%%%%%%%%%%%%%%%%%%%%%%

Ещё одним интересным квантовым процессом, который запрещён в вакууме теоремой Фарри, является расщепление фотона 
в магнитном поле, $\gamma\to\gamma\gamma$. Он также имеет длительную историю исследования, начавшуюся с пионерских 
работ~\cite{Adler:1970,Adler:1971}. Влияние магнитного поля и плазмы на кинематику процесса и, следовательно, на 
соотношение вероятностей различных поляризационных каналов было рассмотрено многочислеными авторами (см. работу~\cite{RChSt:2012} 
и процитированные там статьи). Ещё одним фактором, способным изменить кинематику и амплитуды, что было продемонстрировано 
в~\cite{AM12} на примере радиационного распада нейтрино в сильном поле, является влияние позитрония, которое 
приводит к существенным отклонениям дисперсионного соотношения для фотона от вакуумного в окрестности циклотронного 
резонанса~\cite{ShabUsov1986}.

%%%%%%%%%%%%%%%%%%%%%%%%%%%%%%%%%%%%%%%%%%%%%%%%%%%%%%%%%%%%%%%%%%%%%%%%%%%%%%%%%%%%%%%%%%%%%%%%%%%%%%%%%%%%%%%%%%%%%%%%%%%%%%%%

Настоящая диссертация состоит из введения, трёх глав, заключения, трёх приложений и списка литературы. В первой главе 
вычисляются амплитуды для обобщённого процесса рассеяния $jf \to j^{\, \prime} f^{\prime}$ во внешнем магнитном поле и в замагниченной среде в 
древесном приближении с учётом возможности резонанса на виртуальном электроне. В главе 2 результаты предыдущей главы 
используются для расчёта характеристик конкретных двухвершинных процессов в присутствии замагниченной плазмы в резонансном случае. 
В ней получены выражения для нейтринной светимости процесса $e\gamma\to e\nu\bar\nu$ в плотной замагниченной среде и 
вычислены коэффициент поглощения и сечение рассеяния для комптоновского рассеяния, $e\gamma\to e\gamma$. Глава 3 посвящена 
рассмотрению влияния позитрония на расщепление фотона, $\gamma\to\gamma\gamma$, в сильном магнитном поле; в ней получены 
модифицированные правила отбора по поляризациям и парциальные вероятности процесса в каждом разрешённом канале.




\newpage

{\it Основные обозначения, используемые в диссертации}

\bigskip

Используется 4-метрика с сигнатурой (+ -- -- --), 
а также естественная система единиц $\hbar = 1, c = 1, k_B = 1$.

Элементарный заряд: $e = |e|$, заряд фермиона: $e_f$.

Тензор внешнего поля: $F_{\alpha \beta}$, 
дуальный тензор: 
${\tilde F}_{\alpha \beta} = \frac{1}{2} \ee_{\alpha \beta
\mu \nu} F^{\mu \nu}$.

Обезразмеренный тензор магнитного поля: 
$\varphi_{\alpha \beta} =  F_{\alpha \beta} /B$,  
дуальный обезразмеренный тензор:
${\tilde \varphi}_{\alpha \beta} = \frac{1}{2} \ee_{\alpha \beta
\mu \nu} \varphi^{\mu \nu}$.

У 4-векторов и тензоров, стоящих внутри круглых скобок, тензорные индексы 
полагаются свернутыми последовательно, например:
%
$$(p F F p) = p^{\alpha} F_{\alpha 
\beta} F^{\beta \delta} p_{\delta}; \qquad
(F F p)_\alpha = F_{\alpha 
\beta} F^{\beta \delta} p_{\delta}; \qquad
(F F) = F_{\alpha \beta} F^{\beta \alpha}.$$

Безразмерные тензоры
$\Lambda_{\alpha \beta} = (\varphi \varphi)_{\alpha \beta}$,\,  
$\widetilde \Lambda_{\alpha \beta} = 
(\tilde \varphi \tilde \varphi)_{\alpha \beta}$ связаны соотношением
$\widetilde \Lambda_{\alpha \beta} - \Lambda_{\alpha \beta} = 
g_{\alpha \beta}$.
 
В системе отсчета, где имеется только магнитное поле $\bf B$, направленное 
вдоль третьей оси, 4-векторы с индексами $\bot$ и $\parallel$ относятся 
к подпространствам Евклида \{1, 2\} и Минковского \{0, 3\} соответственно.
При этом 
%
$$\Lambda_{\alpha \beta} = \mbox{diag}(0, 1, 1, 0), \qquad 
\widetilde \Lambda_{\alpha \beta} = \mbox{diag}(1, 0, 0, -1).$$
% 
Для произвольных векторов $p_\mu$, $q_\mu$ имеем:
%
$$p_{\mprp}^\mu = (0, p_1, p_2, 0), \qquad p_{\mprl}^\mu = (p_0, 0, 0, p_3),$$
%
$$(pq)_{\mprp} = (p \Lambda q) =  p_1 q_1 + p_2 q_2, \qquad 
(pq)_{\mprl} = (p \widetilde \Lambda q) = p_0 q_0 - p_3 q_3.$$  

Остальные обозначения те же, что приняты в книге~\cite{landau3}.
