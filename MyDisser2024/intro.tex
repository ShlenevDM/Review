\intro

%
% Используемые далее команды определяются в файле common.tex.
%

% Актуальность работы
\actualitysection
\actualitytext

\developmentsection
\developmenttext

% Цели и задачи диссертационной работы
\objectivesection
\objectivetext

% Научная новизна
\noveltysection
\noveltytext

% Теоретическая и практическая значимость
\valuesection
\valuetext

\methodssection
\methodstext

% Результаты и положения, выносимые на защиту
\resultssection
\resultstext

% Степень достоверности и апробация результатов
\approbationsection
\approbationtext

% Публикации
\pubsection
\pubtext

% Личный вклад автора
\contribsection
\contribtext

% Структура и объем диссертации
\structsection
\structtext

\newpage

Основные обозначения, используемые в диссертации, соответствуют обозначениям, принятым в работе~\cite{KM_Book_2013}.

Используется 4-метрика с сигнатурой (+ -- -- --), 
а также естественная система единиц $\hbar=c=k_B = 1$.

Элементарный заряд: $e = |e_f|$, заряд фермиона: $e_f$. Масса фермиона: $m_f$, масса электрона: $m$. Постоянная тонкой структуры: $\alpha$, 
константа Ферми: $G_F$.

Тензор внешнего поля: $F_{\alpha \beta}$, 
дуальный тензор: 
${\tilde F}_{\alpha \beta} = \frac{1}{2} \ee_{\alpha \beta
\mu \nu} F^{\mu \nu}$.

Обезразмеренный тензор внешнего магнитного поля: 
$\varphi_{\alpha \beta} =  F_{\alpha \beta} /B$,  
дуальный обезразмеренный тензор:
${\tilde \varphi}_{\alpha \beta} = \frac{1}{2} \ee_{\alpha \beta
\mu \nu} \varphi^{\mu \nu}$.

У 4-векторов и тензоров, стоящих внутри круглых скобок, тензорные индексы 
полагаются свернутыми последовательно, например:
%
$$(p F F p) = p^{\alpha} F_{\alpha 
\beta} F^{\beta \delta} p_{\delta}; \qquad
(F F p)_\alpha = F_{\alpha 
\beta} F^{\beta \delta} p_{\delta}; \qquad
(F F) = F_{\alpha \beta} F^{\beta \alpha}.$$

Безразмерные тензоры
$\Lambda_{\alpha \beta} = (\varphi \varphi)_{\alpha \beta}$,\,  
$\widetilde \Lambda_{\alpha \beta} = 
(\tilde \varphi \tilde \varphi)_{\alpha \beta}$ связаны соотношением
$\widetilde \Lambda_{\alpha \beta} - \Lambda_{\alpha \beta} = 
g_{\alpha \beta}$.
 
В системе отсчета, где имеется только магнитное поле $\bf B$, направленное 
вдоль третьей оси, 4-векторы с индексами $\bot$ и $\parallel$ относятся 
к подпространствам Евклида \{1, 2\} и Минковского \{0, 3\} соответственно.
При этом 
%
$$\Lambda_{\alpha \beta} = \mbox{diag}(0, 1, 1, 0), \qquad 
\widetilde \Lambda_{\alpha \beta} = \mbox{diag}(1, 0, 0, -1).$$
% 
Для произвольных векторов $p_\mu$, $q_\mu$ имеем:
%
$$p_{\mprp}^\mu = (0, p_1, p_2, 0), \qquad p_{\mprl}^\mu = (p_0, 0, 0, p_3),$$
%
$$(pq)_{\mprp} = (p \Lambda q) =  p_1 q_1 + p_2 q_2, \qquad 
(pq)_{\mprl} = (p \widetilde \Lambda q) = p_0 q_0 - p_3 q_3.$$  

Остальные обозначения те же, что приняты в книге~\cite{Berestetskii:1989}.