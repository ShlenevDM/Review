\subsection{Фотонейтринный процесс}
Перейдем теперь к рассмотрению другого класса процессов, в которых может реализовываться резонанс на виртуальном электроне. Одним из таких процессов является фоторождение пары нейтрино-антинейтрино на электроне, $e\gamma\to e\nu\bar\nu$, называемое фотонейтринным процессом. Наряду с другими реакциями, в которых нейтринная пара находится в конечном состоянии, интерес к нему вызван тем, что данный процесс может играть определяющую роль в остывании нейтронных звезд. Это связано с тем, что в нейтронных звездах замагниченная плазма прозрачна для нейтрино при значениях параметров (плотности и температуры), которые дают все существующие модели внутреннего строения.

Фотонейтринный процесс был 
впервые рассмотрен в работах~\cite{Ritus:1961,Chiu:1961}, в которых было 
вычислено сечение рассеяния  в 
нерелятивистском 
приближении. Нейтринная светимость, т.е. энергия, уносимая нейтринной парой из 
единичного объема за единицу времени, за счет него была вычислена в 
работах~\cite{Beaudet:1967,Dicus:1972,Munakata:1985,Shindler:1987,Itoh:1989,Itoh:1996,Skobelev:2000}.
 В частности, в~\cite{Itoh:1989,Itoh:1996} были получены таблицы с большим 
количеством данных для фотонейтринной излучательной способности и аналитические 
аппроксимации для них. При этом вклад реакции в процесс нейтринного остывания 
нейтронной звезды полагался пренебрежимо малым, как отмечают авторы 
обзора~\cite{Yakovlev2001}, поскольку в холодной и плотной плазме распад 
плазмона, $\gamma\to\nu\bar\nu$, является намного более эффективным каналом по 
уносу энергии за счет нейтринных пар.

На следующем этапе фотонейтринный процесс изучался с учетом влияния внешней активной среды на дисперсионные и поляризационные свойства фотона~\cite{RCh:2008,Borisov:2012,RumChMik}. Хотя в этих работах были получены уточнения к нейтринной светимости процесса, общий вывод о том, что фотонейтринный процесс является поправкой более высокого порядка к распаду плазмона, остался неизменным.

Наконец, в~\cite{Chistyakov:2014cga,qfthep2017} был рассмотрен резонанс в фотонейтринном процессе. Как было показано в этих работах, даже в магнитарных полях условие~(\ref{bigB}), при котором магнитное поле является доминирующим параметром среды, перестает выполняться при высоких значениях плотности плазмы $\rho \gtrsim 10^8$ г/см$^3$. Такая плотность может достигаться в границе между внешней и внутренней корой магнитара. В результате у электронов (позитронов) плазмы начинают возбуждаться высшие уровни Ландау, что приводит к возможности резонанса в этой реакции. 
Это может увеличивать нейтринную светимость (количество энергии, уносимое нейтринными парами из единицы объема вещества за единицу времени) за счет фотонейтринного процесса в 2-4 раза, чего, однако, недостаточно для того, чтобы он мог конкурировать по эффективности с распадом плазмона.