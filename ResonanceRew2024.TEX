\documentclass[12pt]{article}
% проверка Repository
\usepackage{ifpdf}
\ifpdf
\usepackage{cmap}
\pdfcompresslevel=9
\usepackage[pdftex]{color, graphicx}
\usepackage{epstopdf}
\usepackage[pdftex, colorlinks=false, pdfstartview=FitH, hypertexnames=false, unicode, bookmarks=false]{hyperref} 
\else
\usepackage[dvips]{color, graphicx}
\usepackage[hypertex, colorlinks=false, unicode]{hyperref}
\fi

\usepackage{amssymb, amsmath, amsxtra}
\usepackage{cite}

\usepackage{indentfirst}

\usepackage[T2A]{fontenc}
\usepackage[utf8]{inputenc}
\usepackage[english,russian]{babel}
% \ifpdf\usepackage{epstopdf}\fi
\usepackage{bm}
\usepackage{amsmath,amssymb,amsbsy} 
\usepackage{comment}
\def\mprp{\mbox{\tiny $\bot$}}
\def\mprl{\mbox{\tiny $\|$}}
\def\th{\mbox{th}}

\usepackage{graphicx}
\graphicspath{{Pics/}}
\def\beq{\begin{eqnarray}}
\def\eeq{\end{eqnarray}}
\def\ee{\varepsilon}
\def\lm{\lambda}
\def\ff{\Lambda}
\def\tff{\widetilde \Lambda}
\def\1{1 \to 1 \, 2}
\def\2{1 \to 2 \, 2}

\def\beq{\begin{eqnarray}}
\def\eeq{\end{eqnarray}}
\def\ee{\varepsilon}
\def\lm{\lambda}

\newcommand{\bs}{\boldsymbol} 
\newcommand{\prp}[1]{#1_{\mbox{\tiny $\bot$}}}
\newcommand{\prl}[1]{#1_{\mbox{\tiny $\|$}}}


\newcommand{\ii}{\mathrm{i}}
\newcommand{\dd}{\mathrm{d}}
\newcommand{\eee}{\mathrm{e}} 
% Номера страниц сверху и по центру
%\def\headfont{\small}
%\pagestyle{headcenter}
%\chapterpagestyle{empty}

% Использовать полужирное начертание для векторов
\let\vec=\mathbf

\textheight 230mm  
\textwidth 170mm
\topmargin -5mm
\oddsidemargin 5mm

\pagestyle{plain}

\title{Резонансные комптоноподобные процессы в сильной замагниченной среде (следует подумать)}
\author{Д.А.~Румянцев$^{*}$, Д.М.~Шленев$^{**}$
А.А.~Ярков$^{***}$
\\
{\it Ярославский государственный университет им. П.Г. Демидова, Россия}}

\date{}

\begin{document}
\large
\maketitle
\def\abstractname{\empty}
\baselineskip=22pt

\begin{abstract}

\baselineskip=20pt

{\large В работе рассмотрены различные квантовые процессы с учетом резонанса на виртуальном фермионе.}
\end{abstract}

{\def\thefootnote{*}
\footnotetext{E-mail:  rda@uniyar.ac.ru}
\def\thefootnote{**}
\footnotetext{E-mail: }
\def\thefootnote{***}
\footnotetext{E-mail: a12l@mail.ru}}

\newpage
\unitlength 1mm

\section{Введение}
\section{Резонансные эффекты}
	В сильном магнитном поле поперечная составляющая импульса электрона квантуется. В таком случае энергия электрона определяется так называемым уровнем Ландау $n$ и проекцией импульса вдоль магнитного поля $p_z$:
	\begin{equation}
		E_n = \sqrt{1+p_z^2+2\beta n},
	\end{equation}
	где $\beta=B/B_{кр}$, а $B_{кр} = m_e^2 c^3/e\hbar$. С другой стороны проекция импульса вдоль постоянного магнитного поля, направленного по оси $z$, меняется непрерывно. В связи с квантованностью энергетических состояний, в квантовых процессах могут наблюдаться резонансы, связанные с переходами электрона между уровнями Ландау. Отметим некоторые процессы, в которых возможны резонансы.
	
{\bf Одновершинные процессы}, будучи в замагниченной среде, становятся кинематически разрешены. Особенностью данных процессов можно отметить жесткие кинематические ограничения. Одним из таких процессов является \textit{процесс рождения фотона} $e\to e\gamma $, также называемый циклотронным или синхотронным излучением.  При высоких магнитных полях излучение обусловлено переходами на более низкие уровни Ландау. Следует отметить, 
что при очень сильных магнитных полях, когда $\beta\sim0.2$, электроны, находящиеся на более высоких уровнях Ландау, переходят непосредственно на основной уровень, а не на соседний. С другой стороны обратный к процессу рождения фотона \textit{процесс поглощения фотона} $e\gamma \to e$  приводит к переходу электрона на высшие уровни Ландау. Другой немаловажный квантовый процесс является \textit{Процесс однофотонного рождения электрон-позит\-ронной пары} $\gamma \to e^+e^-$. Особенностью данного процесса является то, что фотон эффективно распадается вблизи точек циклотронного резонанса, где поляризационный оператор фотона имеет сингулярности. 
Однако является подавленным в области ниже порога рождения $\hbar \omega = 2mc^2$.
		
{\bf Двухвершинные процессы.} Типичным примером двухвершинного процесса является {\bfseries комптоновский процесс}  $e \gamma \to e\gamma$. Вблизи циклотронных резонансов сечение комптоновского рассеяния, без учета 
конечной ширины поголощения электрона, становится бесконечным. В таком случае промежуточный (виртуальный) электрон становится реальным, т.е. его закон дисперсии соответствует массовой поверхности, 
а комптоновский процесс становится одновершинным процессом. Поэтому такие резонансы также называются резонансами на виртуальном электроне. Таким образом, в любом процессе, в котором содержится виртуальная частица, 
имеются резонансы. Большое магнитное поле может индуцировать новые взаимодействия частиц. Таким образом могут возникать такие фотон-нейтринные процессы, как \textit{конверсия фотона в пару нейтрино-антинейтрино} 
$\gamma \to \nu \bar{\nu}$ или \textit{излучение фотона нейтрино} $\nu \to \nu \gamma$. Такие процессы имеют петлевую диаграмму с двумя виртуальными электронами и вершинами как слабого взаимодействия так и электромагнитного. 
Резонансные эффекты приводят также к увеличению эффективности \textit{фотонейтринного комптоноподобного процесса} $\gamma e \to e\nu \bar{\nu}$, который, наряду с $\gamma \to \nu \bar{\nu}$, играет важную роль
 в остывании нейтронных звезд.
		
{\bf Трехвершинные процессы.} Кроме того в остывании нейтронных звёзд также играет роль трехвершинный	
\textit{процесс двухфотонной аннигиляции} $\gamma\gamma\to\nu \bar{\nu}$. Среди электромагнитных процессов не менее интересным процессом является 
\textit{процесс рождения электрон-позитронной пары} $\gamma e \to e e^+e^-$, который может быть достаточно эффективным для производства $e^+e^-$-плазмы, в то время, как стандартный механизм 
при аккуратном учете дисперсионных свойств фотонов становится невозможным. Данный процесс также интересен тем, что в нем наблюдаются резонансы как на виртуальном электроне, так и на виртуальном фотоне. 
С точки зрения формирования спектра нейтронных звёзд важным является учет трехвершинного \textit{процесса расщепления фотона} $\gamma \to \gamma \gamma$ и \textit{процесс слияния фотонов} 
$\gamma \gamma \to \gamma$, которые выступают, как механизм изменения числа фотонов. При определенных условиях эти процессы могут конкурировать с комптоновским процессом.  \\
	
\section{Заключение}
\newpage
\bibliographystyle{gost705}
\bibliography{Bib}
\end{document}


