Для определения волновых функций фермионов в присутствии внешнего магнитного поля, которые являются решением уравнения Дирака, cуществует  несколько возможных подходов  (см., 
например~\cite{Johnson:1949,Akhiezer:1965,Sokolov:1968,Melrose:1983a,Sokolov:1986,KM_Book_2003,Bhattacharya:2004,Balantsev:2011,KM_Book_2013}).
При этом, волновые функции, введенные Джонсоном и Липпманом~\cite{Johnson:1949} являются одним из наиболее широко используемых (см. например~\cite{Canuto:1975,Harding:1991,Suh:1999,Gonthier:2000,Jones:2010,Melrose:2020}). 
Эти волновые функции определены как собственные состояния обобщенного оператора импульса. Однако такой подход, будучи примененным к описанию волновых функций фермиона, не обладает необходимыми свойствами, например, симметрией между электронными и позитронными состояниями~\cite{Melrose:1983}. Кроме того, использование данных волновых функций оказывается в некоторых случаях некорректным~\cite{Graziani:1993,Gonthier:2014}, в особенности вблизи циклотронных резонансов, так как требуется правильное описание спиновой зависимости конечной ширины распада промежуточного состояния. 
С другой стороны, волновые функции, используемые Соколовым и Терновым~\cite{Sokolov:1986} обладают необходимыми свойствами и корректно описывают сечение, полученное исходя из этих функций, вблизи циклотронных резонансов. Данные функции являются собственными функциями оператора магнитного момента $\hat{\mu}_z$. Следует отметить, что несмотря на то, что обе волновых функции различаются зависимостями от спина, усредненные по спину физические величины совпадают. Более подробный анализ и сравнение этих подходов можно найти, например, в работе~\cite{Gonthier:2014}.